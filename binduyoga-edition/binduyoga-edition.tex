\documentclass[11pt,showtrims]{memoir} 
                                               % Standardgröße  Indica et Tibetica
\setstocksize{220mm}{155mm} 	               % Papiergröße (das theoretisch beschnitten wird)
\settrimmedsize{220mm}{155mm}{*}	       % Größe der Seite 
\settypeblocksize{170mm}{116mm}{*}	       % bedruckter Bereich 
\setlrmargins{18mm}{*}{*}                      % linker und rechter Rand
\setulmargins{*}{*}{1.2}                       % ober und unterer Rand
\setlength{\headheight}{5pt}                             
\checkandfixthelayout[lines]                   % ist nötig zur abschließenden Layout-Abstimmung
\linespread{1.2}                               % zusätzlicher Durchschuß, "leading", also eine Reihe Blei im Bleisatz, (=Zeilenabstand)
\quarkmarks                                    % Schnittmarken 

                                             %%% reledmac setup %%%
\usepackage[series={A,B},noend,noeledsec,nofamiliar,noledgroup]{reledmac}                              %  zwei Fußnotenebenen werden angelegt: A und B
\Xarrangement[A]{paragraph}                                                                            % A als Paragraph formatiert
\Xarrangement[B]{paragraph}                                                                            % B als Paragraph
\lineation{page} \linenummargin{outer}                                                                 % automatische Zeilenzählung seitenweise (nicht für das ganze Dokument) u. am Außenrand      
\newcommand{\var}[3]{\edtext{}{\lemma{\hskip -3.6mm \textrm{#1}~~#2}\Afootnote[nonum]{{#3}}}}          % Varianten für nummerierte Strophen mit manuellem Bezug auf
                                                                                                       % Vers und Pāda mit 3 Klammerpaaren:  \var{1a}{kritischer Text}{Variante}                     
\newcommand{\varc}[2]{\nobreak\hskip 0pt \edtext{}{\lemma{#1}\Afootnote{{#2}}}\hskip 0pt \relax}       % Varianten für Prosatext mit automat. Bezug auf Zeilen 
\newcommand{\varcc}[1]{\edtext{}{\lemma{\hskip -4mm}\Afootnote[nosep]{#1}}}                            % habe vergessen, wozu das war 
\newcommand{\notes}[2]{\edtext{}{\lemma{#1}\Bfootnote{{\footnotesize\textrm{#2}}}}}                    % 2. Apparat für Anmerkungen     \notes{tat}{Pronomen}
                               %%%%%   einprägsamere Macros für reledmac 
\def\prose{\pstart}\def\endprose{\pend}                                                                % einprägsamere Varianten statt dem ständigen \pstart \pend
\def\mula{\pstart}\def\endmula{\pend\medskip}                                                          % Variante mit zusätzlichem Abstand am Ende für mūla-Text 

                                                 %%%%%%   XeTeX- Setup       sollte klar sein
\usepackage{polyglossia,fontspec,xunicode,libertineotf}     % Ich verwende Libertine für die Sigla, sieht besser aus.
 \widowpenalty=10000
 \clubpenalty=10000
  \setdefaultlanguage{english} \setotherlanguage{sanskrit}
 \setmainfont{Adobe Text Pro} %Adobe Text Pro in Mac umstellen                                                       
\newfontfamily\devanagarifont[Script=Devanagari,Mapping=RomDev,Scale=1.2,LetterSpace=1.1]{Murty Sanskrit}
\usepackage{microtype}
\lineskiplimit=-3pt     % sonst werden die Fußnotezeilen bei Nagari auseinandergeschoben


                                              %   OTHER MACROS  -- Beispiele               --- das ist alles spezifisch für bestimmte Texte
 \def\colophon#1{\centerline{{\small #1}}}
\def\fourbijas{oṃ aiṃ hrīṃ śrīṃ}
\def\thirtysevenbijas{37}                                 \def\etc{°~}
\def\title#1{\centerline{\large{\devanagarifont{#1}}}\thispagestyle{empty}}
\def\prakarana#1{\centerline{[#1]}}
\def\lem{\textrm{]\ }}
\def\coni{\textrm{\footnotesize coni.}}\def\conj{\textrm{coni.}}
\def\coniI{\textrm{\footnotesize coni. Isaacson}}
\def\trick{\textrm{}}                                            \def\smallcirc{$\cdot$}
\def\bijaAanusvara{aṃ}
\def\ornament{\centerline{\aldineleft}}   
\def\ornamentb{{\footnotesize\centerline{\ding{105}}}\leavevmode\smallskip}
\def\ornamentstar{{\footnotesize\centerline{\ding{105}}}\leavevmode\smallskip}
% \def\ornamentstar{\centerline{\Huge{\schmuckfont{s}}}}
\def\amlig{a{ṃ}  }  % das war ein Fehler, so geht es. 
\def\crux{$\dag$}

% Grunddefinitionen für die einzelnen Siglen. Die Buchstaben werden mit LibertineDisplay gesetzt.
% Ich habe mit zwei Varianten der Definition gespielt, die erste mit \superscript und \subscript,
% die zweite ohne (es funktionieren beide. 

%\def\acpc#1#2#3{\begin{english}\libertineDisplay #1\end{english}%
%  \textsuperscript{\begin{english}\kern-1.8pt \tiny{#3}\end{english}}%
%  \textsubscript{\begin{english}\kern-1em \tiny{#2}\end{english}}}
%\def\sigl#1#2{\begin{english}\libertineDisplay #1\end{english}%
%  \textsubscript{\begin{english}\kern-.2em \tiny{#2}\end{english}}}

\def\acpc#1#2#3{\begin{english}\libertineDisplay #1\end{english}%
  \raisebox{.8ex}{{\begin{english}\kern-1.8pt \tiny{#3}\end{english}}}%
  \raisebox{-.3ex}{{\begin{english}\kern-.8em \tiny{#2}\end{english}}}}
\def\sigl#1#2{\begin{english}\libertineDisplay #1\end{english}%
     \raisebox{-.3ex}{{\begin{english}\kern-.2em \tiny{#2}\end{english}}}}

\def\illeg{\textrm{\footnotesize{illegible}\space}}
\def\recto{\lower-.8ex\hbox{\scriptsize{r}}}
\def\verso{\lower-.8ex\hbox{\scriptsize{v}}}
\def\om{\textrm{\footnotesize om.\ }}
\def\emend{\textrm{\footnotesize em.\ }}
\def\conj{\textrm{\footnotesize conj.\ }}
\def\acpc#1#2#3{\begin{english}\libertineDisplay #1\end{english}\textsuperscript{\begin{english}\kern-1.8pt #3\end{english}}\textsubscript{\begin{english}\kern-6pt #2\end{english}}}
\def\sigl#1#2{\begin{english}\libertineDisplay #1\end{english}\textsubscript{\begin{english}\kern-1.2pt #2\end{english}}}


%%%%%%%%%%%%%%                                    Tattvabinduyoga                              %%%%%%%%%%%%%%%%%%%

\def\edprint{{\sigl{E}{1}}} \def\edprintac{\acpc{E}{1}{ac}\,} \def\edprintpc{\acpc{E}{1}{pc}\,}                 % Druckausgabe
\def\nepal{{\sigl{N}{1}}} \def\nepalac{\acpc{N}{1}{ac}\,} \def\nepalpc{\acpc{N}{1}{pc}\,}                       % NGMPP B 38-31
\def\pune{{\sigl{P}{1}}} \def\puneac{\acpc{P}{1}{ac}\,} \def\punepc{\acpc{P}{1}{pc}\,}                          % Pune BORI 664
\def\oxford{{\sigl{B}{1}}} \def\oxfordac{\acpc{B}{1}{ac}\,} \def\oxfordpc{\acpc{B}{1}{pc}\,}                    % Bodleian Oxford D 4587
\def\lalchand{{\sigl{L}{1}}} \def\lalchandac{\acpc{L}{1}{ac}\,} \def\lalchandpc{\acpc{L}{1}{pc}\,}              % Lalchand Research Library LRL5876
\def\unbekannt{{\sigl{U}{1}}} \def\unbekanntac{\acpc{L}{1}{ac}\,} \def\unbekanntpc{\acpc{L}{1}{pc}\,}           % Birch irgendwas   

%%% Local Variables:
%%% mode: latex
%%% TeX-master: t
%%% End:
     %   Die Siglen für den Apparat in eigener Datei
   
\chapterstyle{crosshead}
\nouppercaseheads
\pagestyle{headings}
\createmark{section}{both}{nonumber}{}{}
\def\englishnote#1{\begin{english}\textrm{#1}\end{english}}

\usepackage{microtype} %besserer Zeilenumbruch

%große Frage: Reihenfolge der Zeugen im Apparat? 

\begin{document}
\chapter{Conventions in the Critical Apparatus}
\section{Sigla in the Critical Apparatus}
\begin{itemize}
\item \edprint : Printed Edition
\item \pune : Pune BORI 664
\item \lalchand : Lalchand Research Library LRL5876
\item \oxford : Bodleian Oxford D 4587
\item \nepal : NGMPP B 38-31
\item \dehlia: IGNCA 30019
\item \dehlib: IGNCA 30020
\item \ujjaina: SORI 1574
\item \ujjainb: SORI 6082
\end{itemize}

The order of the readings in the critical apparatus is arranged according to the quality of readings in decending order.  

\section{Punctuation}

The very inconsistent use of punctuation marks in the witnesses at hand makes standardization necessary. Deviation of punctuation marks will not be documented in the critical apparatus. The usual standard conventions are followed:

Especially in the verse poetry, a \textit{daṇḍa} marks the end of a half verse, half of the \textit{śloka}, and the double \textit{daṇḍa} marks the end of a verse. A half verse is a \textit{pāda}, at least in some literary works, this is concluded by a \textit{daṇḍa} and the end of a \textit{śloka} by a double \textit{daṇḍa}. In the prose the single \textit{daṇḍa} indicates the end of a sentence and the double \textit{daṇḍa} marks the end of a paragraph.

\section{Sandhi}

Among the witnesses we see deviating and inconsistent application of \textit{sandhi}. There is no clear evidence that originally \textit{sandhi} was intentionally not applied. This edition will therefore apply \textit{sandhi} consistently throughout the constituted text to provide a readable text sticking to contemporary conventions in Sanskrit. To simplyfy the apparatus the variant readings concerning \textit{sandhi} are not recorded to the most part. Exceptions are made in remarkable individual cases. 

\section{Class Nasals}

Again, due to inconsistent use of class nasals among the witnesses \textit{anusvāra}s have been substituted with the respective class nasals throughout the critical edition. To simplyfy the apparatus deviating usage of class nasals is not documented in the apparatus. 

%\section{Notes}
%\begin{itemize}
%\item probelm with the strongly deviating structuration of the sentences between the two groups of witnesses 1. \edprint \pune \lalchand \oxford und 2. \nepal \dehlia: Even though it is not entirely clear which of the group is closer to the supposedly original of the author, as an hypothesis I suspect that given the cummulative evidence of a) \nepal and \dehlia generally have the better, more consistent and meaningful readings, b) are in several cases more extensive in their descriptions and c) generally there seems to be stemmatic hints pointing towards the inferiority of group 1., currently I decided to prefer the structuration of sentences of group 2.
%\item it is clear that the author of \edprint has not understood the specific tantric-haṭhayogic terminology. This is evident by his ambitions to emend the terminology of the śakta-pīṭhas, which are very important in the haṭhayogic traditions. He takes uḍḍīyāna as upayana and jālaṃdhāra as jatyadana etc.
%\item in many cases the manuscripts do not utilize \textit{sandhi}
%\item Notiz für mich selbst: ENTSCHEIDEN OB SANDHI SETZEN ODER NICHT! HANDSCHRIFTEN VS. EDPRINT! DURCHGÄNGIG BISHER IMMER SANDHI BIS ZUM 1/3 DES NAVAMACAKRAABSCHNITTES
%\item jetzt klar: die prakriyā-Formen sind charakteristisch und der Sandhi wird wahrscheinlich bewusst im Prosa ausgelassen!
%\item in cakra 9 unbedingt die Verweise für das siebzehnte kalā ergänzen! Jürgen fragen nach e-texts?!
%\item kurzen Abschnitt zum Titel verfassen und diskutieren 
%\end{itemize}

\begin{sanskrit} 
\chapter{Critical Edition of the \textit{Yogatattvabindu}}
  
\title{Yogatattvabindu}\markboth{\devanagarifont Yogatattvabindu}{\devanagarifont Yogatattvabindu}
\beginnumbering
\bigskip

\pstart
\centerline{\begin{english}\textrm{\small{[Introduction]}}\end{english}}
\pend
\bigskip
\sloppy %besserer Zeilenumbruch
\prose
\noindent 
%śrī gaṇeśāya namaḥ /                                                    rājayogāntargataḥ //  binduyogaḥ   \edprint 
%śrī gaṇeśāya namaḥ /                                                    atha tattvabiṃduyogaprāraṃbhaḥ    \lalchand
%śrī ṇe ya maḥ /                                                         atha rājayoga         liṣyate      \pune
%śrī gaṇeśāya namaḥ // śrī gurave namaḥ //                               atha rājayogaprakāro  likhyate //  \nepal
%śrī gaṇeśāya namaḥ // śrī sarasvatyai namaḥ // śrī nirañjanāya namaḥ // atha rājayogaprakāro  likhyate //  \dehlia
%śrī gaṇeśāya namaḥ / oṃ śrī niraṃjanāya //                              atha rājayogaprakāra  likhyate //  \ujjaina
%śrī gaṇeśāya namaḥ /                                                    atha rājayoga         likhyate //  \ujjainb
% 
śrī gaṇeśāya namaḥ\varc{śrī gaṇeśāya namaḥ}{śrī ṇe ya maḥ \pune śrī gaṇeśāya namaḥ // śrī gurave namaḥ // \nepal śrīgaṇeśāya namaḥ // śrī sarasvatyai namaḥ // śrī nirañjanāya namaḥ // \dehlia} // atha rājayogaprakāro likhyate //\varc{atha rājayogaprakāro likhyate}{atha rājayogaprakāra likhyate \ujjaina atha rājayoga likhyate \ujjainb atha rājayoga liṣyate \pune rājayogāntargataḥ // binduyogaḥ \edprint atha tattvabiṃduyogaprāraṃbhaḥ \lalchand} 
\endprose
%Homage to Śrī Gaṇeśa. Now the methods of rājayoga are laid down.
%-------------------------
\prose
\noindent 
% \om                       \edprint
% \om                       \lalchand
% \om                       \oxford
%rājayogasyedaṃ phalaṃ      \pune
%rājayogasya idaṃ phalaṃ    \nepal
%rājayogasya idaṃ phalaṃ // \dehlia
%rājayogasya idaṃ phalaṃ    \ujjaina
%rājayogasyedaṃ phalaṃ /    \ujjainb
%
rājayogasyedaṃ phalaṃ \varc{rājayogasyedaṃ}{rājayogasya idaṃ \nepal \dehlia \ujjaina}/
%
%this is the result of rājayoga:
%
%-------------------------
%
% \om                                                                                                                                                                \edprint
% \om                                                                                                                                                                \lalchand
% \om                                                                                                                                                                \oxford
%yena rājayogenānekarājyabhogasamaya   eva    anekapārthivavinodaprekṣaṇasamaya  eva    bahutarakālaṃ śarīrasthitirbhavati    sa eva  rājayogaḥ tasyaite     bhedāḥ      \pune
%yena rājayogenānekarājyabhogasamaya   eva /  anekapārthivavinodaprekṣaṇasamaya  eva /  bahutarakālaṃ śarīrasthitirbhavati    sa eva  rājayogaḥ /  tasya ete bhedāḥ /  \nepal
%yena rājayogena anekarājyabhogasamaya eva // anekapārthivavinodaprekṣaṇasamaya  eva // bahutarakālaṃ śarīrasthitirbhavati // sa eva  rājayogaḥ // tasya ete bhedāḥ / \dehlia
%yena rājayogena anekarājyabhogasamaya eva // anekapārthivavinodaprekṣaṇasamaya  eva // bahutarakālaṃ śarīrasthitirbhavati    sa evaṃ rājayogaḥ    tasya ete bhedāḥ //   \ujjaina 
%yena rājayogena anekarājyabhogasamaya eva // anekapārthivavinodaprekṣyaṇasamaya eva // bahutarakālaṃ śarīrasthitirbhavati // sa eva  rājayogastaisyaite     bhedāḥ //   \ujjainb
yena rājayogenānekarājyabhogasamaya eva / anekapārthivavinodaprekṣaṇasamaya\varc{°prekṣaṇasamaya}{°prekṣyaṇasamaya \ujjainb} eva / bahutarakālaṃ śarīrasthitirbhavati / sa eva\varc{eva}{evaṃ \ujjainb} rājayogaḥ\varc{rājayogaḥ}{rājayogas° \ujjainb} / tasyaite bhedāḥ\varc{tasyaite bhedāḥ}{tasya ete bhedāḥ \nepal \dehlia \ujjaina} /
%
%Rājayoga is that by which longterm durability of the body arises even amongst manifold royal pleasures even amongst the manifold royal entertainments and spectacle. This truly is rājayoga. Of this [rājayoga] these are the varieties:
%-------------------------
%
% \om                                                                                                                                                                \edprint
% \om                                                                                                                                                                \lalchand
% \om                                                                                                                                                                \oxford
% kriyāyogaḥ 1 jñānayogaḥ 2 caryāyogaḥ 3 haṭhayogaḥ 4 karmayogaḥ 5 layayogaḥ 6 dhyānayogaḥ 7 maṃtrayogaḥ 8 lakṣyayogaḥ 9 vāsanāyogaḥ 10 śivayogaḥ 11 brahmayogaḥ 12 advaitayogaḥ 13 siddhayogaḥ 14 rājayogaḥ 15 ete paṃcadaśayogāḥ \pune
% kriyāyogaḥ / jñānayogaḥ / caryāyogaḥ / haṭhayogaḥ / karmayogaḥ / layayogaḥ / dhyānayogaḥ / maṃtrayogaḥ / lakṣyayogaḥ / vāsanāyogaḥ / śivayogaḥ / brahmayogaḥ / advaitayogaḥ / rājayogaḥ / siddhayogaḥ / ete paṃcadaśayogāḥ // \nepal
% kriyāyogaḥ // jñānayogaḥ // caryāyogaḥ // haṭhayogaḥ // karmayogaḥ // layayogaḥ // dhyānayogaḥ // maṃtrayogaḥ // lakṣyayogaḥ // vāsanāyogaḥ // śivayogaḥ // brahmayogaḥ // advaitayogaḥ // rājayogaḥ // siddhayogaḥ // ete paṃcadaśayogāḥ // \dehlia
% kriyāyogaḥ // jñānayogaḥ // tvaryāyogaḥ // haṭhayogaḥ // karmayogaḥ // layayogaḥ // dhyānayogaḥ maṃtrayogaḥ  lakṣayogaḥ  vāsanāyogaḥ  śivayogaḥ  brahmayogaḥ  advaitayogaḥ  rājayogaḥ  siddhayogaḥ ete paṃcadaśayogāḥ  \ujjaina
% kriyāyogaḥ // jñānayogaḥ // caryāyogaḥ // haṭhayogaḥ // karmayogaḥ // nayayogaḥ // dhyānayogaḥ // maṃtrayogaḥ // lakṣyayogaḥ // vāsanāyogaḥ // śivayogaḥ // brahmayogaḥ // advaitayogaḥ // siddhayogaḥ // rājayogaḥ // evaṃ paṃcadaśāyogā bhavaṃti // \ujjainb
%
%
%
kriyāyogaḥ / jñānayogaḥ / caryāyogaḥ\varc{caryāyogaḥ}{tvaryāyogaḥ \ujjaina} / haṭhayogaḥ / karmayogaḥ / layayogaḥ\varc{layayoga}{nayayogaḥ \ujjainb} / dhyānayogaḥ / mantrayogaḥ / lakṣyayogaḥ\varc{lakṣyayogaḥ}{lakṣayoga \ujjaina} / vāsanāyogaḥ / śivayogaḥ / brahmayogaḥ / advaitayogaḥ / siddhayogaḥ / rājayogaḥ\varc{siddhayogaḥ / rājayogaḥ}{rājayogaḥ / siddhayogaḥ \nepal \dehlia \ujjaina}\notes{kriyāyogaḥ ... rājayogaḥ}{\englishnote{\small The initial codification of 15 \textit{yoga}s appears in \nepal, \pune, \dehlia, \ujjaina and \ujjainb. It is ommitted in \edprint \lalchand. \oxford can't be determined due to missing folios. \pune is the only witness which numbers the \textit{yoga}s with \textit{devanāgarī}-digits. The other witnesses separate the list with single or double \textit{daṇḍa}s.}} / ete pañcadaśayogāḥ\varc{ete pañcadaśayogāḥ}{evaṃ paṃcadaśāyogā bhavaṃti \ujjainb} //
%
\endprose
\pstart
\centerline{\begin{english}\textrm{\small{[Description of \textit{kriyāyoga}]}}\end{english}}
\pend
\bigskip
\prose
\noindent
% \om                                     \edprint
% \om                                     \lalchand
% \om                                     \oxford
%idānīṃ kriyāyogasya lakṣaṇaṃ kathyate /  \pune
%idānīṃ kriyāyogasya lakṣaṇaṃ kathyate /  \nepal
%idānīṃ kriyāyogasya lakṣaṇaṃ kathayate / \dehlia
%idānīṃ kriyāyogasya lakṣaṇaṃ kathyate /  \nepal
%idānīṃ kriyāyogasya lakṣaṇaṃ kathyate /  \ujjaina
%atha   kriyāyogas   lakṣaṇaṃ          // \ujjainb
%
idānīṃ\varc{idāṇīṃ}{atha \ujjainb} kriyāyogasya\varc{kriyāyogasya}{kriyāyogas \ujjainb} lakṣaṇaṃ kathyate\varc{kathyate}{kathayate \dehlia \om \ujjainb}/
%
%Now the characteristic of the Yoga of [mental] action (kriyāyoga) described.
%
%-------------------------
%
\endprose
\medskip
%
%\AtEveryStanza{\vskip 2mm}??? Er will das nicht beim kompilieren. Warum?
\setlength{\stanzaindentbase}{13pt}
\setstanzaindents{0,4,4,4,4}
%
%
% \om                                                   \edprint
% \om                                                   \lalchand
% \om                                                   \oxford
%kriyāmuktir    ayaṃ yogaḥ    svapiṇḍe siddhidāyakaḥ    \pune
%kriyāmuktir    ayaṃ yogaḥ /  svapiṇḍe siddhidāyakaḥ /  \nepal 
%kriyāmuktir    ayaṃ yogaḥ    svapiṇḍe siddhidāyakaḥ /  \dehlia
%kriyāyuktir    ayaṃ yogaḥ /  svapiṇḍe siddhidāyakaḥ /  \ujjaina
%kriyāmuktiḥ // ayaṃ yogaḥ    svapiṃ?  siddhidāyakaṃ // \ujjainb 
%
\stanza
kriyāmuktirayaṃ yogaḥ svapiṇḍe siddhidāyakaḥ/ &
%
%This Yoga is liberation through [mental] action. It bestows (siddhi) in ones own body
%
%-------------------------
% \om                                                   \edprint
% \om                                                   \lalchand
% \om                                                   \oxford
%yaṃ yaṃ karoti kallolaṃ kāryāraṃbhe manaḥ sadā         \pune
%yaṃ yaṃ karoti kallolaṃ kāryāraṃbhe manaḥ sadā/        \nepal
%yaṃ yaṃ karoti kallolaṃ kāryāraṃbhe manaḥ sadā/        \dehlia 
%yaṃ yaṃ karoti kallolaṃ kāryāraṃbhe manaḥ sadā/ 1      \ujjaina
%yaṃ yaṃ karoti kallolaṃ kāryāraṃbhe manaḥ sadā/        \ujjainb
%
yaṃ yaṃ karoti kallolaṃ kāryāraṃbhe manaḥ sadā/ &
%
%Each wave the mind creates at the beginning of an action,
%
%%-------------------------
% \om                                                        \edprint
% \om                                                        \lalchand
% \om                                                        \oxford
%   tattataḥ kuñcanaṃ kurvan kriyāyogas tato bhavet           \pune
%   tattataḥ kuñcanaṃ kurvan kriyāyogas ato bhava    //      \nepal
%   tattataḥ kuñcanaṃ kurvan kriyāyogas ato bhava    //      \dehlia 
%   taṃkṛ taṃ kuñcanaṃ kurvan kriyāyoga sato ?va     //1//   \ujjaina
%   tatastataḥ kuṃcanaṃ kurvan kriyāyogas tato bhavet //1//  \ujjainb
%
tattataḥ\varc{tattataḥ}{tatastataḥ \ujjainb taṃkṛ taṃ \ujjaina} kuñcanaṃ kurvan kriyāyogastato bhavet\varc{bhavet}{ato bhava \nepal \dehlia sato va \ujjaina}// \& 
\medskip
%
%of all those one shall withdraw oneself. Then kriyāyoga arises.
%
%-------------------------
% \om                                                                                                  \oxford
% \om                                                                                                  \lalchand
%kṣamā vivekaṃ vairāgyaṃ śāntiḥ santoṣaniṣpṛhā    etadyuktiyuto yogī         kriyāyogī nigadyate       \edprint
%kṣamāvivekavairāgyaṃ    śāntiḥ santoṣanispṛhā    etat yuktiyuto yogī        kriyāyogī nigadyate       \nepal
%kṣamāvivekavairāgyaṃ    śāntiḥ santoṣanispṛhaḥ   etat yuktiyuto yogī        kriyāyogī nigadyate       \dehlia
%kṣamāvivekavairāgyaṃ    śāntiḥ santoṣanispṛhāḥ   etadyuktiyuto yogī         kriyāyogī nigadyate       \pune
%kṣamāvivekavairāgya---- śāntisantoṣaniḥspṛhī     etadyuktiyuto yosau        kriyāyogī nigadyate       \ujjaina 
%kṣamā vivekaṃ vairāgyaṃ śāntisaṃtoṣaniṣpṛhāḥ //  etatmuktiyuto yogī         kriyāyogī nigadyate //2// \ujjainb
%
%The text of the Printed Edition starts here ---> 
%
\stanza 
kṣamāvivekavairāgyaṃ\varc{°vivekavairāgyaṃ}{vivekaṃ vairāgyaṃ \edprint \ujjainb} \notes{kṣamā°}{\englishnote{\small \edprint starts here.}} śāntiḥ\varc{śāntiḥ}{śānti° \ujjaina \ujjainb} santoṣanispṛhāḥ\varc{santoṣanispṛhāḥ}{saṃtoṣaniṣpṛhāḥ \ujjainb °santoṣaniṣpṛhā \edprint santoṣanispṛhaḥ \dehlia santoṣanispṛhā \nepal °santoṣaniḥspṛhī \ujjaina } / &
yuktiyuto yogī\varc{°yuktiyuto}{°yuktiyuto yosau \ujjaina}{muktiyuto yogī} kriyāyogī nigadyate//\&
\medskip
%
%Patience, discrimination, equanimity, peace, modesty, desireless: The yogī who is endowed with these means is said to be a kriyāyogī.
%
%------------------------
% \om                                             \oxford
% \om                                             \lalchand
% mātsaryaṃ mamatā māyā hiṃsā ca   madagarvitā /  \edprint
% mātsarya  mamatā māyā hiṃsāśā    madagarvitāḥ    \pune
% mātsarya  mamatā māyā hiṃsāḥ //  madagarvatā /  \nepal    -> the hiṃsā---''ḥ//'' in \nepal looks like a śā -> indicator that the others copied from \nepal? 
% mātsarya  mamatā māyā hiṃsāśā    madagarvatā /  \dehlia
% mātsaryaṃ mamatā māyā hiṃsāśā    madagarvatā /  \ujjaina
% mātsaryaṃ mamatā māyā hiṃsāśā    madagarvatā /  \ujjainb
%
%
\stanza 
mātsaryaṃ\varc{mātsaryaṃ}{mātsarya \nepal \dehlia \pune} mamatā māyā hiṃsā ca\varc{hiṃsā ca}{hiṃsāḥ \nepal hiṃsāśā \pune \dehlia \ujjaina \ujjainb} madagarvatā\varc{madagarvatā}{madagarvitā \edprint madagarvitāḥ \pune}/ & 
kāmakrodhabhayaṃ \varc{kāmakrodhabhayaṃ \pune \nepal \edprint}{krāmakrodho bhayaṃ \dehlia}lajjā lobhamohau tathāśuciḥ\varc{tathā 'śuciḥ \pune \dehlia \nepal}{tathā śuciḥ \edprint}// \& %müsste es eigentlichn nicht tathāśuciḥ sein?
\medskip
%
%Envy, selfishness, cheating, violence and intoxication,
%
%-----------------------
%  \om                                                       \oxford
%  atha dveṣo ghṛṇālasyaṃ bhrāṃtir   daṃbho kṣamā bhramaḥ //     \lalchand
%  rāgadveṣau ghṛṇālasyaṃ bhrāntitvaṃ     mokṣamā bhramaḥ /      \edprint
%  rāgadveṣau ghṛṇālasyaṃ bhrāṃtir   ddaṃbhokaṣmā bhramaḥ        \pune
%  rāgadveṣau ghṛṇālasyaṃ bhrāṃtir   daṃbho kṣamā bhramaḥ //4// \nepal   
%  rāgadveṣau ghṛṇālasyaṃ bhrāṃtir   debho  kṣamā bhramaḥ //     \dehlia
%  rāgadoṣau  ghṛṇālasyaṃ bhrāṃti    daṃbha kṣamī bhramaḥ 4    \ujjaina
%  rāgadveṣau ghṛṇālasyaṃ bhrāṃtir   daṃbho kṣamā bhramaḥ //    \ujjainb
%
%
\stanza 
rāgadveṣau\varc{rāgadveṣau}{rāgadoṣau \ujjaina athadveṣo \lalchand}\notes{rāga°}{\englishnote{\small \lalchand starts here.}} ghṛṇālasyaṃ bhrantirdaṃbho \varc{bhraṃtirdaṃbho}{bhrāṃtirdebho \dehlia bhrāntitvaṃ \edprint bhrāṃti daṃbha \ujjaina}kṣamā bhramaḥ\varc{kṣamā bhramaḥ}{mokṣamābhramaḥ \edprint °kṣamī bhramaḥ \ujjaina}/ &
%
%Attachment and aversion, indignation and idleness, impatience and dizzyness:
%
%-----------------------
%
%
%  \om                                             \oxford
%  yasyai tāni na vidyaṃte kriyāyogī sa ucyate //   \lalchand
%  yasyai tāni ca vidyante kriyāyogī sa ucyate 3    \edprint
%  yasyai tāni na vidyaṃte kriyāyogī sa ucyate      \pune
%  yasyai tāni na vidyaṃte kriyāyogī sa ucyate //   \nepal   
%  yasyai tāni na vidyaṃte kriyāyogī sa ucyate //   \dehlia
%  yasyai tāni na vidyaṃte kriyāyogī sa ucyate      \ujjaina
%  yasyai tāni na vidyaṃte kriyāyogī sa ucyate //4//\ujjainb
%
%
yasyaitāni na \varc{na}{ca \edprint}vidyante kriyāyogī sa ucyate// \&
\medskip
%
%Whoever does not possess these is called a kriyāyogī.
%
%%-----------------------
%
%
\prose
\noindent
%  \om                                                                                          \oxford
%  yasyāntaḥkaraṇe kṣamāvivekavairāgyaśāntisantoṣādīny                        utpadyante //     \edprint
%  yasyāṃtaḥkaraṇe kṣamāvivekavairāgyaśāṃtisaṃtoṣa         ityādīny           utpādyaṃte        \pune
%  tasyāṃtaḥkaraṇe kṣamāvivekavairāgyaśāṃtisaṃtoṣa         ityādīnotpādyaṃte                    \lalchand
%  yasyāṃtaḥkaraṇe kṣamāḥ vivekavairāgya / śāṃtisaṃtoṣa    ityādīni           utpādyaṃte        \nepal   
%  yasyāṃtaḥkaraṇe kṣamā // vivekavairāgya // śāṃtisaṃtoṣa ityādīni           utpādyaṃte //     \dehlia
%  yasyāṃtaḥkaraṇe kṣamāvivekavairāgyaśāṃtisaṃtoṣa         ityādīna niraṃtaram   utyaṃte        \ujjaina
%  yasyāṃtaḥkaraṇe kṣamāvivekavairāgyaśāṃtisaṃtoṣa         ityādayoniraṃtaraṃ utpādyaṃte        \ujjainb
%
%
yasyāntaḥkaraṇe kṣamāvivekavairāgyaśāntisantoṣa ityādīnyutpādyaṃte\varc{kṣamāvivekavairāgyaśāntisantoṣa ityādīnyutpādyaṃte}{kṣamāḥ vivekavairāgya / śāṃtisaṃtoṣa ityādīni utpādyaṃte \nepal kṣamā // vivekavairāgya // śāṃtisaṃtoṣa ityādīni utpādyaṃte // \dehlia kṣamāvivekavairāgyaśāṃtisaṃtoṣa ityādīnotpādyaṃte \lalchand kṣamāvivekavairāgyaśāntisantoṣādīnyutpadyante \edprint kṣamāvivekavairāgyaśāṃtisaṃtoṣa ityādīna niraṃtaram utyaṃte \ujjaina yasyāṃtaḥkaraṇe kṣamāvivekavairāgyaśāṃtisaṃtoṣa ityādayoniraṃtaraṃ utpādyaṃte \ujjainb}
%
%[When] patience, discrimination, equanimity, peace, contentment etc. are generated in his mind, 
% 
%-----------------------
%
% \om \oxford
%  sa eva bahukriyāyogī kathyate /      \edprint
%  sa eva bahukriyāyogī kathyate        \pune
%  sa eva bahukriyāyogī kathyate //     \lalchand
%  sa eva bahukriyāyogī kathyate /      \nepal
%  sa eva bahukriyāyogā sa kathyate //  \dehlia
%  sa eva bahukriyāyogī kathyate /      \ujjaina
%  sa eva bahukriyāyogī tkacyate /      \ujjainb
%
%
sa eva bahukriyāyogī kathyate\varc{kathyate}{sa kathyate \dehlia tkacyate \ujjainb} /
%
%Then [then] alone he is called a yogī of many actions (bahukriyāyogī). 
% 
%-----------------------
%
%
% \om \oxford
%                kāpaṭyaṃ      vittaṃ   hiṃsā    tṛṣṇā    mātsaryam    ahaṃkāraḥ    roṣaḥ kṣayaṃ   lajjālobhamohā      aśucitvaṃ                       pākhaṃḍatvaṃ       bhrāntiḥ indriyavikāraḥ kāmaḥ          ete yasya manasi pratidinaṃ vyunā bhavanti /  \edprint
%                kāpaṭyaṃ      vittaṃ   hiṃsā    tṛṣṇā    mātsaryaṃ    ahaṃkāraḥ    roṣo bhayaṃ    lajjā lobhaḥ mohaḥ  aśucitvaṃ rāgaḥdveṣaḥ   ālasyaṃ pākhaṃḍitvaṃ       bhrāṃtiḥ indriyaṃ vikāraḥ kāmaḥ        ete yasya manasi pratidinaṃ nyunā bhavanti   \pune
%                kāpayaṃ     //vitaṃ // hiṃsā // tṛṣṇā // mātsaryaṃ // ahaṃkāraḥ // roṣo bhayaṃ // lajjālobhaḥ // moha aśucitvaṃ // rājadveṣa  alasyaṃ // pākhaṃḍitvaṃ // bhrāṃtiḥ // itivikāraḥ // kāmaḥ        eta yasya manasi pratidinaṃ nyunā bhavaṃti//\lalchand
% yasyāṃtakaraṇe kapatyaṃ māyā vitvaṃ   hiṃsā    tṛṣṇā    mātsaryaṃ    ahaṃkāraḥ    roṣobhayaṃ     lajjā // lobhamohā  asucitvaṃ rāgadveṣaḥ // alasyaṃ pāṣaṃḍitvaṃ        bhraṃtiḥ / iṃdriyaivikāraḥ / kāmaḥ     ete yasya manasi pratidinaṃ nyunā bhavaīti / \nepal
%                kāpaṭyaṃ māya vitvaṃ   hiṃsā    tṛṣṇā    mātsarya     ahaṃkāraḥ    roṣobhayaṃ     lajjā // lobhamohā  asucitvaṃ rāgadveṣaḥ // ālasyaṃ pāṣaṃḍitvaṃ        bhraṃtiḥ // iṃdriyavikāraḥ // kāmaḥ // ete yasya manasi pratidinaṃ nyunā bhavaṃti //  \dehlia
%                kāpachaṃ yāya vitvaṃ   hiṃsā    tṛṣṇā    mātsarya     ahaṃkāraḥ    roṣaḥ bhayaṃ   lajā lobhamohā      aśucitvaṃ rāgadveṣaḥ    ālasyaṃ pākhaṃḍitvaṃ       bhraṃtiḥ iṃdriyavīkāraḥ    kāmaḥ       rāte yasya manasi pratidinaṃ nyunā bhavaṃti //      \ujjaina
%                kāpaṭyaṃ pāpā titaṃ    hiṃsā    tṛṣṇā    mātsaryaṃ // ahaṃkāraḥ    roṣobhayaṃ     lajjā ----mohā      aśucitvaṃ rāgadveṣaḥ    ālasyaṃ pākhaṃḍitvaṃ //    bhraṃtiḥ iṃdriyavikāraḥ //-----        etate yasya manasi pratidinaṃ nyunā bhavaṃti // \ujjainb
%
%
\varc{yasyāntaḥkaraṇe \nepal}{\om \dehlia \edprint \pune \lalchand \ujjaina \ujjainb} kāpaṭyaṃ \varc{kāpaṭyaṃ}{kapatyaṃ \nepal kāpayaṃ \lalchand kāpachaṃ \ujjaina} māyā\varc{māyā}{yāya \ujjaina pāpā \ujjainb \om \edprint \pune \lalchand} vittaṃ \varc{vittaṃ}{vitaṃ \lalchand vitvaṃ \nepal \dehlia \ujjaina titaṃ \ujjainb} hiṃsā tṛṣṇā mātsaryamahaṃkāraḥ\varc{mātsaryam}{mātsaryaṃ \pune \lalchand \nepal \ujjainb mātsarya \dehlia \ujjaina} roṣobhayaṃ\varc{roṣobhayaṃ}{roṣaḥ bhayaṃ \ujjaina roṣaḥ kṣayaṃ \edprint} lajjā\varc{lajjā}{lajā \ujjaina} lobho moho 'śucitvaṃ\varc{\conj lobho moho 'śucitvaṃ}{lajjālobhamohā aśucitvaṃ \edprint \ujjaina lajjā lobhaḥ mohaḥ aśucitvaṃ \pune  lajjālobhaḥ // moha aśucitvaṃ // \lalchand lajjā // lobhamohā  asucitvaṃ \nepal \dehlia lajjā mohā \ujjainb} rāgadveṣau \varc{\conj rāgadveṣau}{rāgadveṣaḥ \nepal \dehlia \lalchand \ujjaina \ujjainb \om \edprint}\notes{rāgadveṣau}{\englishnote{\small All witnesses currently at hand read \textit{rāgadveṣaḥ}. I conjectured to \textit{rāgadveṣau}. However, there is the possibility to also read \textit{rāgo dveṣaḥ}. Another problem that comes up here is the very inconsistent use of \textit{daṇḍa} in the witnesses within the given lists. I seems impossible to determine which variant is the most reliable. As in the other cases I decided to get rid of \textit{daṇḍa} and apply \textit{sandhi} consistently.}}ālasyaṃ\varc{ālasyaṃ}{\om \pune} pākhaṃḍitvaṃ\varc{pākhaṃḍitvaṃ}{pāṣaṃḍitvaṃ \dehlia \nepal pākhaṃḍatvaṃ \edprint}bhrāntirindriyavikāraḥ\varc{\emend bhrāntirindriyavikāraḥ}{brāntiḥ indriyavikāraḥ \edprint bhraṃtiḥ / iṃdriyaivikāraḥ / \nepal \dehlia bhraṃtiḥ iṃdriyavikāraḥ // \ujjainb bhraṃtiḥ iṃdriyavīkāraḥ \ujjaina bhrāṃti indriyaṃ vikāraḥ \pune bhrāṃtiḥ // itivikāraḥ // \lalchand}kāmaḥ\varc{\om \ujjainb} ete \varc{ete}{eta \lalchand rāte \ujjaina etate \ujjainb} yasya manasi pradidinaṃ nyūna bhavanti\varc{bhavanti}{bhavīti \nepal} /
%
%Fraud, illusion, property,violence, craving, envy, ego, anger, anxiety, shame, greed, error, impurity, attachment, aversion, idleness, heterodoxy, false view, affection of the senses, sexual desire: He who diminishes these from day to day in is mind,...
%
%-----------------------
%sa eva bahukriyāyogī kathyate // \edprint
%sa eva bahukriyāyogī kathyate // \pune
%sa eva bahukriyāyogī kathyate // \lalchand
%sa eva bahukriyāyogī kathyate // \nepal
%sa eva bahukiyāyogī kathyate // \dehlia
%sa eva bahukiyāyogī kathyaṃte // \ujjaina 
sa eva bahukriyāyogī kathyate\varc{kathyate}{kathyaṃte \ujjaina} // 
%
%
%he alone is called a yogī of many actions (bahukriyāyogī).
%
%-----------------------
\endprose

\bigskip
\pstart
\centerline{\begin{english}\textrm{\small{[Varieties of \textit{rājayoga}: Siddhakuṇḍalinīyoga and Mantrayoga]}}\end{english}}
\pend
\bigskip

\prose
% \om                                   \oxford
%idānīṃ rājayogasya bhedāḥ kathyante // \edprint
%idānīṃ rājayogasya bhedāḥ kathyaṃte    \pune
%idānīṃ rājayogasya bhedāḥ              \lalchand
%idānīṃ rājayogasya bhedāḥ kathyaṃte    \nepal
%idānīṃ rājayogasya bhedāḥ kathyaṃte // \dehlia     
% \om                                   \ujjaina
%idānīṃ rājayogasya bhedāḥ kathyaṃte // \ujjainb
%
%
idānīṃ rājayogasya bhedāḥ kathyante \varc{kathyante}{\om \lalchand}\notes{idānīṃ rājayogasya bhedāḥ kathyante //}{\englishnote{\small The whole sentence is omitted in \ujjaina.}}/
%
%Now varieties of \textit{rājayoga} will be described.
%
%-----------------------
%
%te ke \edprint
%te ke \pune
%te ke \lalchand
%ke te // \dehlia
%ke te / \nepal 
%ke te \ujjaina
%te ke \ujjainb
%
%
ke te\varc{ke te \nepal \dehlia \ujjaina}{te ke \pune \lalchand \edprint \ujjainb} /
%
%Which are these?
%
%-----------------------
% \om                                      \oxford
%ekaḥ siddhakuṇḍalinīyogaḥ / mantrayogaḥ / \edprint
%ekaḥ siddhakuṃḍaṃliṃ yogaḥ maṃtrayogaḥ    \pune
%ekaḥ siddhakuṇḍalanīyoga /                \lalchand
%ekaḥ siddhakuṇḍalinīyogaḥ maṃtrayogaḥ /   \nepal
%ekaḥ siddhakuṃḍalanīyogaḥ mantrayogaḥ //  \dehlia 
%ekaḥ siddhakuṇḍaliniyogaḥ mantrayogaḥ     \ujjaina
%ekaḥ siddhakuṇḍalinīyoga // mantrayogaḥ     \ujjainb
%
%
ekaḥ siddhakuṇḍalinīyogaḥ \varc{siddhakuṇḍalinīyogaḥ \dehlia}{siddhakuṇḍalanīyogaḥ \nepal \pune \edprint siddhakuṇḍalanīyoga \lalchand} mantrayogaḥ\notes{mantrayogaḥ}{\englishnote{\small The sudden appearance of \textit{mantrayoga} seems very odd. Esspecially considering that the content that follows in this section doesn't mention the practice of mantra at all. It might me a mistake, or a later insertion.}}/
%
%One is siddhakuṇḍalinīyoga [and one] is mantrayoga.
%
%-----------------------
% \om                         \oxford
%astu rājayogaḥ kathyate /    \edprint
%amū rājayogau kathyete       \pune
%amū rājayogau kathyate //    \lalchand
%amū rājayogau kathyate       \nepal
%amū rājayogau kathyate //    \dehlia 
%amū rājayogau kathyate       \ujjaina
%amū rājayogau kathyaṃte //   \ujjainb
%
amū\varc{amū}{astu \edprint} rājayogau\varc{rājayogau}{rājayogaḥ \edprint} kathyete\varc{kathyete}{kathyate \nepal \dehlia \lalchand \ujjaina \edprint kathyaṃte \ujjainb} / 
%
%These two rājayogas are described [in the following].
%
%-----------------------
% \om                                                            \oxford
%mūlakandasthāne    ekā tejorūpā    mahānāḍī varttate /            \edprint
%mūlaṃ kaṃdasthāne  ekā tejorūpā    mahānāḍī varttate            \pune
%mūlakaṃdasthāne    ekā tejorūpā    mahānāḍī vartate               \lalchand
%mūlakaṃdasthāne    eka tejorūpā    mahānāḍī varttate /            \nepal
%mūlakaṃdasthāne    ekā tejorūpā    mahānāḍī varttate //           \dehlia 
%mūlakaṃdasthāne    ekā tejorūpā    mahānāḍī vartate /             \ujjaina
%mūlakaṃdasthāne // ekā tejorūpā // mahānāḍī pravarttate /   \ujjainb
%
mūlakandasthāne \varc{mūlakandasthāne}{mūlaṃ kaṃdasthāne \pune} ekā\varc{ekā}{eka \nepal} tejorūpā mahānāḍī vartate\varc{vartate}{pravarttate \ujjainb} /
%
%
%At the location of the root-bulb exists one major vessel in the form of energy (\textit{tejas}). 
%
%-----------------------
% \om                                                      \oxford
%iyamekanāḍī /  iḍāpiṃgalāsuṣumṇā      etān bhedān prāpnoti /    \edprint
%iyaṃ ekanāḍī   iḍāpiṃgalāsuṣumṇā      etān bhedān prāpnoti       \pune
%trayaṃ kā nāḍī iḍāpiṃgalāsuṣumnā //   etān bhedān prāpnoti  \lalchand
%iyaṃ ekā nāḍī  iḍāpiṃgalāsuṣumnān /   ete  bhedān prāpnoti   \nepal
%iyaṃ ekā nāḍī  iḍāpiṃgalasuṣumnān //  ete  bhedān prāpnoti   \dehlia 
%iyaṃ ekā nāḍī  iḍāpiṃgalāsuṣumnā      etān bhedān prāpnoti     \ujjaina
%iyaṃ eka nāḍī  iḍāpiṃgalāsuṣumṇā      etān bhegān prāpnoti     \ujjainb
%
iyaṃ \varc{iyaṃ}{trayaṃ \lalchand} ekā \varc{ekā}{eka \pune \edprint \ujjainb kā \lalchand} nāḍī iḍāpiṅgalāsuṣumnān \varc{°suṣumnān}{suṣumnā \lalchand \ujjaina suṣumṇā \pune \edprint \ujjainb}etān\varc{etān}{ete \nepal \dehlia}bhedān\varc{bhedān}{bhegān \ujjainb} prāpnoti /
%
%
%
%This single vessel reaches into these openings which are \textit{iḍā}, \textit{piṅgalā} and \textit{suṣumnā}.
%
%-----------------------
%
% \om                                                      \oxford
%vāmabhāge candrarūpā iḍā nāḍī varttate /      \edprint
%vāmabhāge caṃdrarūpā iḍā nāḍī varttate       \pune
%vāmabhāge caṃdrarūpā iḍā nāḍī varttate //     \lalchand
%vāmabhāge caṃdrarūpā iḍā nāḍī varttate /      \nepal
%vāmabhāge caṃdrarūpā iḍā nāḍī varttate /      \dehlia 
%vāmabhāge caṃdrarūpā iḍā nāḍī vartate         \ujjaina
%vāmabhāge caṃdrarūpā     nāḍī pravarttate //  \ujjainb
%
%
vāmabhāge candrarūpā iḍā\varc{iḍā}{\om \ujjainb} nāḍī vartate\varc{vartate}{pravarttate \ujjainb} / 
%
%On the left side is the iḍā-channel, being a resemblence of the moon. 
%
%-----------------------
%
% \om                                                      \oxford
%dakṣiṇabhāge sūryarūpā piṅgalā  nāḍī    varttate /  \edprint
%dakṣiṇabhāge sūryarūpā piṃgalā  nāḍī    varttate    \pune
%dakṣiṇabhāge sūryarūpā piṃgalā  nāḍī    varttate // \lalchand
%dakṣiṇabhāge sūryarūpā piṃgalā  nāḍī    varttate // \nepal
%dakṣiṇabhāge sūryarūpā piṃgalā  nāḍī    varttate //   \dehlia 
%dakṣiṇe bhāge sūryarūpā piṃgalā nāḍī    vartate      \ujjaina
%dakṣiṇabhāge sūryarūpā piṃgalā  nāḍī pravartate //  \ujjainb
%
%
dakṣiṇabhāge\varc{dakṣiṇa°}{dakṣiṇe \ujjainb} sūryarūpā piṅgalā nāḍī vartate\varc{vartate}{pravarttate \ujjainb} /
%
%On the right side exists the piṅgalā-channel, being a resemblence of the sun. 
%
%
%-----------------------
%
% \om                                                                  \oxford
%madhyamārge `tisūkṣmā padminī taṃtusamākārā  koṭividyutsamaprabhā      \edprint
%madhyamārge `tisūkṣmā padmanī taṃtusamākāra! koṭividyutsamaprabhā     \pune
%madhyamārge `tisūkṣmā padmanī taṃtusamākārā  koṭividyutsamaprabhā      \lalchand
%madhyamārge atisūkṣmā padmanī taṃtusamākārā  koṭividyutsamaprabhā //   \nepal
%madhyarge   atisūkṣmā padminī taṃtusamākārā  koṭividyutsamaprabhā //   \dehlia 
%madhyamārge atisūkṣmā padminī taṃtusamākārā  koṭividyutsamaprabaḥ      \ujjaina
%madhyamārge  tisūkṣmā padminī taṃtusamākārā  koṭividyutsamaprabhā //    \ujjainb
madhyamārge\varc{madhyamārge}{madhyarge \dehlia}'tisūkṣmā padminī\varc{padminī}{padmanī \nepal \lalchand \pune} tantusamākārā\varc{tantusamākārā}{tantusamākāra \pune} koṭividyutsamaprabhā\varc{°prabhā}{°prabhaḥ \ujjaina} /
%
%Within the middle path is a lotuspond being very subtle. [It is] made from a web of light [and it] shines like a thousand lightnings.
%
%-----------------------
%\om                                                                  \oxford
%bhuktimuktipradā                                     'syā jñānotpattau satyaṃ puruṣaḥ sarvajño  bhavati      idānīṃ suṣumṇāyāṃ jñānotpattāv---upāyāḥ kathyante     \edprint
%bhuktimuktidā                                        asyā jñānotpattau satyāṃ puruṣaḥ sarvajño  bhavati      idānīṃ suṣumṇāyā  jñānotpattau   upāyāḥ kathyaṃte     \pune
%bhuktimuktipradā //                                  asyā jñānotpattau satyāṃ puruṣaḥ sarvajño  bhavati   // idānīṃ suṣumnā    jñānotpattau   upāyaḥ kathyate //    \lalchand
%bhuktimukti--------------------------------------------------dotpanne  sati---puruṣaḥ sarrvajño bhavati    / idānīṃ suṣumnāyāḥ jñanotpanno    'pāyāḥ kathyaṃte //  \nepal
%bhuktimukti--------------------------------------------------dotpanne  sati---puruṣaḥ sarrvajño bhavati    / idānīṃ suṣumnāyāḥ jñanotpattau   upāyāḥ kathyaṃte //  \dehlia 
%bhuktimukti--------------------------------------------------dotpanne  sati---puruṣaḥ sarrvajño bhavati    / idānīṃ  suṣumnāya-jñanotpattau   upāyāḥ kathyaṃte //      \ujjaina
%bhuktimuktidā śivarūpiṇī suṣumṇā nāḍī pravarttate // asyā jñānotpattau satyāṃ puruṣa--sar-vajño bhavati   // idānīṃ suṣumṇāyā  jñānotpattau   upāyā  kathyaṃte // \ujjainb
%
bhuktimuktidā\varc{bhuktimuktidā}{bhuktimuktipradā \edprint \lalchand bhuktimuktidā śivarūpiṇī suṣumṇā nāḍī pravarttate \ujjainb} /
%
%She is the bestower of enjoyment and liberation. 
%
%-----------------------
%
asyāṃ\varc{asyāṃ}{\emend asyā \pune \lalchand \ujjainb \edprint \om \nepal \dehlia \ujjaina} jñānotpattau\varc{jñānotpattau}{utpanne \dehlia \nepal \ujjaina} satyāṃ\varc{satyāṃ}{satyaṃ \edprint sati \nepal \dehlia \ujjaina} puruṣaḥ sarvajño\varc{puruṣaḥ sarvajño}{puruṣasarvajño \ujjainb} bhavati / idānīṃ suṣumṇāyāṃ\varc{suṣumṇāyāṃ}{suṣumnāyāḥ \nepal \dehlia suṣumṇāyā \pune \ujjainb suṣumnāya \ujjaina suṣumṇā \lalchand} jñānotpattāvupāyāḥ\varc{jñānotpattavupāyāḥ}{jñānotpanno 'pāyāḥ \nepal °upāyaḥ \lalchand °upāyā \ujjainb} kathyante\varc{kathyante}{kathyate \lalchand} //\notes{jñānotpattāvupāyaḥ}{\englishnote{\small It is not clear if the list given at the beginning of the text codifying the fifteen \textit{yoga}s belongs to the original text or was a later addition by a another hand. One primary reason for this suspicion is that the structure of the \textit{yoga}s in the text does not equal the list. The text begins with a description of \textit{kriyāyoga} and continues to describe \textit{siddhakuṇḍaliniyoga} and somewhat suprisingly mentions \textit{mantrayoga} in the same breath. One starts wondering why the structure of the text does not follow the codification. However the mention of \textit{jñānotpattau upāyaḥ} might be a clue why the second \textit{yoga} in the list might be \textit{jñānayoga}. So far it seems to me that there are three options or a combination of these to explain these apparent inconsistencies: 1. The text is highly corrupted. 2. The codification was a later addition of another hand. 3. The term \textit{jñānayoga} is listed due to the results of \textit{siddhakuṇḍalinīyoga}, which is the generation of knowledge due to the practice of a certain \textit{yoga} involving the central channel, as mentioned in this section of the text.}}
%
% She is the bestower of enjoyment and liberation. While abiding in (\textit{satyāṃ}) her (\textit{asyāṃ}) knowledge arises [to the point of which] the person becomes all-knowing. The means for the genesis of knowledge in the central channel \textit{suṣumnā} will now be described.
%
%
\endprose
\bigskip
\pstart
\centerline{\begin{english}\textrm{\small{[Description of the first Cakra]}}\end{english}}
\pend
\bigskip
\prose
%
%
%\om                                       \oxford
%ādau caturdalaṃ mūlaṃ cakraṃ varttate /   \edprint
%ādau caturddalaṃ mūlaṃ cakraṃ varttate /   \pune
%ādau caturdalamūlacakraṃ varttate //    \lalchand
%ādau caturdalaṃ mūlacakraṃ varttate   \nepal
%ādau caturdalaṃ mūlacakraṃ varttate    \dehlia 
%ādau caturdalaṃ mūlaṃ cakraṃ vartate    \ujjaina
%ādau caturdalaṃ mūlacakraṃ pravarttate //   \ujjainb
%
%
ādau caturdalaṃ mūlacakraṃ\varc{caturdalaṃ mūlacakraṃ \nepal \dehlia \ujjainb}{caturdalaṃ mūlaṃ cakraṃ \edprint \pune \ujjaina caturdalamūlacakraṃ \lalchand} vartate\varc{vartate}{pravarttate \ujjainb} /
%
%At the beginning [of the central channel?] exists the root-cakra having four petals.
%-----------------------
%
%
%\om                                       \oxford
%prathamādhāracakraṃ varttate / gudāsthānaṃ    raktavarṇaṃ    gaṇeśadaivataṃ    siddhibuddhiśaktimuṣakavāhanam       kurmaṛṣiḥ /  ākuṃcamudrā /    apānavāyuḥ                                   caturdaleṣu     rajaḥsattvatamomanāṃsi /  vaṃ śaṃ ṣaṃ saṃ    madhyatrikoṇe triśikhāt    tanmadhye trikoṇākāraṃ kāmapīthaṃ varttate//    \edprint
%prathamaṃ ādhāracakraṃ         gudāsthānaṃ    raktavarṇaṃ    gaṇeśāṃ daivataṃ  siddhibuddhiśaktir mukhako vāhanam   kurmaṛṣiḥ    ākuṃcanamudrā    apānavāyuś-----------------------------------caturddaleṣu    rajaḥsattvatamomanāṃsi    vaṃ śaṃ ṣaṃ saṃ    madhyatrikoṇe triśikhā     tanmadhye trikoṇākāraṃ kāmapīthaṃ varttate //   \pune
%prathamaṃ ādhāracakraṃ         gudāsthānaṃ    raktavarṇaṃ    gaṇeśadaivataṃ    siddhibuddhiśaktimuṣako vāhanaṃ //   kurmaṛṣiḥ    ākuṃcanamudrā    apānavāyuḥ                                   caturddaleṣu    rajaḥsattvatamomanāṃsi // vaṃ śaṃ ṣaṃ saṃ    madhyatrikoṇe triśikhā     tanmadhyatrikoṇākāraṃ kāmapīthaṃ vartate        \lalchand
%prathamaṃ ādhāracakraṃ         gudāsthānaṃ // raktavarṇaṃ // gaṇeśadaivataṃ // siddhibuddhiśaktiḥ muṣako vāhanaṃ // kurmaṛṣiḥ // ākuṃcanamudrā // apānavāyu // umīrkalā // ojakhinīdhāraṇā // caturddaleṣu // rajaḥsattvatamomanāṃsi //  vaṃ śaṃ ṣaṃ saṃ // madhyatrikoṇe trirekhā //  tanmadhye trikoṇākāraṃ kāmapīthaṃ varttate //   \ujjainb    
%---------------------------------------------------------------------------------------------------------------------------------------------------------------------------------------------------------------------------------------------------------------------------------------tanmadhyatrikoṇākāraṃ kāmapiṭhaṃ varttate /   \nepal
%---------------------------------------------------------------------------------------------------------------------------------------------------------------------------------------------------------------------------------------------------------------------------------------tanmadhye trikoṇākāraṃ kāmapiṭhaṃ varttate /  \dehlia 
%---------------------------------------------------------------------------------------------------------------------------------------------------------------------------------------------------------------------------------------------------------------------------------------tanmadhye trikoṇākāraṃ kāmapiṭhaṃ varttate /   \ujjaina
%
prathamaṃ ādhāracakraṃ\varc{prathamaṃ ādhāracakraṃ}{prathamādhāracakraṃ varttate \edprint} gudāsthānaṃ raktavarṇaṃ gaṇeśadaivataṃ\varc{gaṇeśa°}{gaṇeśāṃ \pune} siddhibuddhiśaktiṃ mūṣako vāhanaṃ \varc{siddhibuddhiśaktiṃ muṣako vāhanaṃ}{\emend siddhibuddhiśaktir mukhako vāhanaṃ \pune siddhibuddhiśaktiḥ muṣako vāhanaṃ \ujjainb siddhibuddhiśaktimūṣako vāhanaṃ \lalchand siddhibuddhiśaktimuṣakavāhanam \edprint} kurmaṛṣiḥ ākuñcanamudrā\varc{ākuñcanamudrā}{ākuṃcamudrā \edprint}apānavāyuścaturdaleṣu\varc{°vāyuścaturdaleṣu}{°vāyuḥ caturddaleṣu \edprint \lalchand °vāyu // umīrkalā // ojakhinīdhāraṇā // caturddaleṣu // \ujjainb} rajaḥsattvatamomanāṃsi vaṃ śaṃ ṣaṃ saṃ madhyatrikoṇe triśikhā\varc{triśikhā}{triśikhāt \edprint trirekhā \ujjainb} tanmadhye\varc{tanmadhye}{tanmadhya° \lalchand \nepal} trikoṇākāraṃ kāmapiṭhaṃ vartate/\notes{prathamaṃ ... triśikhā}{\englishnote{\small The whole section is missing in \nepal,\dehlia and \ujjaina.}}
%
%
%The first cakra of support (ādhāra) is at the anus, [it] is red-colored, [it] has Gaṇeśa as its deity, [he] is success, intelligence and power, [and has] a rat as [his] mount, the Ṛṣi [of it] is Kūrma, [its mudrā] is the mudrā of contraction (ākuñcamudrā), [its] vitalwind is apāna, in the four petals [of it resides] rajas, sattva, tamas and mindstuffs?!(manāṃsi) [symbolized by the syllables] “vaṃ”, “śaṃ”, “ṣaṃ” and “saṃ”, in the middle [of it] is a triangle, in the middle [of the triangle] is a trident, and kāmapīṭha in the shape of a triangle.
%
%-----------------------
%
%
%\om                                       \oxford
%tatpīṭhamadhye 'gniśikhākāraikā mūrtir varttate /   \edprint
%tatpīṭhamadhye magniśikhākārā ekā mūrtir varttate /  \pune
%tatpīṭhamadhye   jniśikhāka!rāṇakā mūrti varttate //   \lalchand
%tatpīṭhamadhye  agniśikhākārā ekā mūrttir varttate //   \nepal
%tatpīṭhamadhye  agniśikhākārā ekā mūrttir varttate //    \dehlia 
%tatpīṭhamadhye  agniśikhākārā ekā mūrttir varttate //   \ujjaina
%tatpīṭhamadhye  agniśikhākārā ekā mūrttirasmi      //   \ujjainb
%
%
tatpīṭhamadhye 'gniśikhākāraikā \varc{agniśikhākāraikā \edprint}{agniśikhākārā ekā \nepal \dehlia magniśikhākārā ekā \pune jñiśikhākaraṇakā \lalchand}mūrtirvarttate /
%
%
%-----------------------
%
%
%\om                                       \oxford
%tasyāḥ mūrtirdhyānakāraṇāt sakalaśāstrakāvyanāṭakādisakalavāṅmayaṃ vinābhyāsena puruṣasya manomadhye sphurati,     \edprint
%tasyā mūrter dhyānakaraṇāt sakalaśāstrakāvyanāṭakādisakalavāṅmayaṃ vinābhyāsena puruṣasya manomadhye sphurati      \pune
%tasyā mūrtir dhyānakaraṇāt sakalaśāstrakāvyanāṭakādi //   vāṅmayaṃ vinābhyāsena puruṣasya manomadhye sphuraṃti!    \lalchand
%tasyāḥ mūrter dhyānakāraṇāt sakalaśāstrakāvyanāṭakādi sakalavāgmayaṃ vinābhyāsena puruṣasya manomadhye sphurati   \nepal
%tasyāḥ mūrter dhyānakāraṇāt sakalaśāstrakāvyanāṭakādi sakalavāgmayaṃ vinābhyāsena puruṣasya manomadhye sphurati    \dehlia 
%tasyā  mūrtair dhyānakāraṇāt sakalaśāstrakāvyanāṭakādi sakalavāgmayaṃ vinābhyāsena puruṣasya manomadhye sphurati   \ujjaina
%tasyā          dhyānakāraṇāt sakalaśāstrakāvyanāṭakādi sakalavāṅmayaṃ vinābhyāsena puruṣasya manomadhye sphurati //  \ujjainb
%
%
tasyāḥ \varc{tasyāḥ \nepal \dehlia \edprint}{tasyā \pune \lalchand} mūrterdhyānakaraṇāt\varc{mūrter \nepal \dehlia \pune \edprint}{mūrti \lalchand}\varc{dhyānakaraṇāt \nepal \dehlia \pune \lalchand}{dhyānakāraṇāt \edprint} sakalaśāstrakāvyanāṭakādisakalavāṅmayaṃ\varc{°nāṭakādisakalavāṅmayaṃ \pune \edprint}{°nāṭakādisakalavāgmayaṃ \nepal \dehlia °nāṭakādi // vāṅmayaṃ \lalchand} vinābhyāsena puruṣasya manomadhye sphurati //
%
%-----------------------
%
%
\endprose
\bigskip
\pstart
\centerline{\begin{english}\textrm{\small{[Description of the second Cakra]}}\end{english}}
\pend
\bigskip
\prose
idānīṃ dvitīyaṃ svādhiṣṭānacakraṃ\notes{svādhiṣṭānacakraṃ}{\englishnote{\small One would rather expect \textit{svādhiṣṭhānacakraṃ}. Instead the witnesses all transmit \textit{svādhiṣṭānacakraṃ}.}} ṣaṭdalaṃ \varc{ṣaṭdalaṃ \nepal \dehlia \pune \lalchand}{ṣaḍdalaṃ \edprint} uḍḍīyāṇapīṭhasaṃjñakaṃ \varc{uḍḍīyāṇapīṭhasaṃjñakaṃ}{\emend uḍḍīyāṇāpīṭhasaṃjñakaṃ \dehlia upāyanapīṭhasaṃjñakaṃ \edprint uḍḍīyāne pīṭhasaṃjñakaṃ \nepal uḍḍīyānapīṭhaṃ saṃjñakaṃ \pune uḍḍīyān pīṭhaṃ saṃjñako \lalchand} bhavati / tanmadhye 'tiraktavarṇam\varc{'tiraktavarṇam \pune \lalchand}{atiraktavarṇaṃ \nepal \dehlia \edprint} / tasya dhyānāt sādhako\varc{sādhako \pune \lalchand \edprint}{sadhakaḥ \nepal \dehlia} 'tisundaro\varc{'tisundaro \pune \lalchand \edprint}{atisundaro \nepal \dehlia} bhavati / yuvatināmativallabho\varc{yuvatināmativallabho \pune \lalchand}{yuvatīnāṃ vallabho \edprint yuvatīnāṃ vallabho \nepal \dehlia} bhavati\varc{bhavati \nepal \dehlia \lalchand \edprint}{na bhavati \pune} / pratidinamāyurvardhate \varc{pratidinam \edprint \pune \lalchand}{dinaṃ dinaṃ prati \nepal dinaṃ prati \dehlia}//
\endprose
\bigskip
\pstart
\centerline{\begin{english}\textrm{\small{[Description of the third Cakra]}}\end{english}}
\pend
\bigskip
\prose
tṛtīyaṃ \varc{tṛtīyaṃ \nepal \dehlia \lalchand}{tṛtīye \edprint \pune}nābhistāne daśadalaṃ\varc{daśadalaṃ \pune \edprint \nepal \dehlia}{daśadala \lalchand} padmaṃ\varc{padmaṃ \edprint \pune}{padme \lalchand padma \nepal \dehlia} varttate / tanmadhye pancakoṇaṃ cakraṃ varttate\varc{tanmadhye paṃcakoṇaṃ cakraṃ varttate \nepal \dehlia \pune \edprint}{\om \lalchand} / tanmadhye ekā mūrtirvarttate\varc{tanmadhye ekā mūrtirvarttate \nepal \dehlia \pune \edprint}{\om \lalchand} / tasyāstejo\varc{tasyās \edprint \pune \lalchand}{tasyāḥ \dehlia tasyā \nepal} jihvayā kathituṃ\varc{kathituṃ \edprint \pune}{kathayituṃ \nepal \dehlia \lalchand} na śakyate/ tasyāḥ \varc{tasyāḥ \nepal \dehlia \edprint}{tasyā \lalchand \pune} mūrterdhyānakāraṇāt \varc{mūrter \edprint \pune \nepal \dehlia}{mūrtir \lalchand}\varc{dhyānakaraṇāt \nepal \dehlia}{dhyānakāraṇāt \edprint \pune \lalchand} puruṣasya śarīraṃ \varc{śarīraṃ \edprint \lalchand \nepal}{śarīrāṃ \dehlia} sthiraṃ bhavati \varc{śarīraṃ sthiraṃ bhavati \edprint \lalchand \nepal \dehlia}{\om \pune}//
\endprose
\bigskip
\pstart
\centerline{\begin{english}\textrm{\small{[Description of the fourth Cakra]}}\end{english}}
\pend
\bigskip
\prose
caturthaṃ\varc{caturthaṃ \edprint \pune \nepal \dehlia}{caturthakaṃ \lalchand} hṛdayamadhye dvadaśadalaṃ kamalaṃ\varc{hṛdayamadhye dvadaśadalaṃ kamalaṃ \edprint \pune \lalchand}{kamalaṃ hṛdayamadhye dvadaśadalaṃ \nepal \dehlia} vartate / atitejomayatvāddṛṣtigocaraṃ\varc{atitejomayatvād \edprint}{atitejomayatvāt \pune \nepal \dehlia atitejomayatvā \lalchand} na bhavati / tanmadhye 'ṣṭadalamadhomukhaṃ\varc{'ṣṭadalam \edprint}{'ṣṭadale \pune 'ṣṭadalaṃ \lalchand aṣṭadalaṃ \nepal \dehlia}\varc{adhomukhaṃ \edprint \nepal \dehlia}{adhomukha \lalchand mukhaṃ \pune}kamalaṃ varttate / tanmadhye prāṇavāyoḥ sthānamaṣṭadalakamalamadhye \varc{sthānam \edprint \nepal \dehlia}{sthānaṃ \pune \lalchand} liṃgākārā karṇikā\varc{karṇikā \edprint \lalchand \nepal \dehlia}{varṇikā \pune} kathyate / tasyāḥ kaliketi \varc{kaliketi \pune \lalchand \nepal \dehlia}{karṇiketi \edprint} saṃjñā \varc{saṃjñā \edprint \pune \nepal \dehlia}{\om \lalchand} tatkalikāmadhye \varc{tatkalikāmadhye \pune \dehlia}{tatakalikāmadhye \nepal tatkarṇikāmadhye \edprint madhye \lalchand} padmarāgaratnasamānavarṇāṃguṣthapramāṇaikā \varc{padmarāgaratnasamānavarṇāṃguṣthapramāṇaikā}{\emend padmarāgasamānavarṇāṃguṣṭhapramāṇaikā \edprint padmarāgasamānavarṇāṃ- guṣṭhapramāṇā ekā \pune madhye padmaratnasamānavarṇā / aṅguṣṭhāpramāṇā / ekā \lalchand padmarāgaratnasamānavarṇā aṅguṣthapramāṇo eka \nepal padmarāgaratnasamānavarṇā aṅguṣthapramāṇāt ekā \dehlia} puttalikā vartate / tasyā \varc{tasyā \edprint \pune}{tasyāḥ \nepal \dehlia tasya \lalchand}jīveti saṃjñā \varc{jīveti saṃjñā}{\emend jīveti saṃjñāḥ \nepal jīveti saṃjña \dehlia jīvasaṃjñā \edprint \pune \om \lalchand}tasyā \varc{tasyā \edprint \pune}{tasyāḥ \nepal \dehlia \om \lalchand}balamatha ca svarūpaṃ \varc{balam atha ca svarūpaṃ \nepal \dehlia}{balaṃ atha svarūpaṃ \pune balamadhyasvarūpaṃ \edprint balasapasvvarūpaṃ \lalchand} koṭijihvābhirvaktuṃ \varc{koṭijihvābhir \edprint \pune \nepal \dehlia}{koṭijihvāyābhi \lalchand}naiva śakyate /
%
%tasyā  jīvasaṃjñā       tasyā balamadhyasvarūpaṃ               koṭijihvābhir vaktuṃ naiva śakyate // \edprint
%tasyā  jīvasaṃjñā       tasyā balam atha svarūpaṃ              koṭijihvābhir vaktuṃ naiva śakyate // \pune 
%tasya                        balasa?pasvarūpaṃ                koṭijihvāyābhi vaktuṃ na śakyate // \lalchand 
%tasyāḥ jīveti saṃjñāḥ  tasyāḥ balaṃ atha ca svarūpaṃ      koṭijihvābhir vaktuṃ na śakyate // \nepal
%tasyāḥ jīveti saṃjña / tasyāḥ balaṃ atha ca svarūpaṃ     koṭijihvābhir vaktuṃ na śakyate // \dehlia
%
%
%Her technical designation is embodied soul. Not even with a thousand tongues it is possible to talk about her nature and her power.
asyā\varc{asyā}{asyā \edprint \pune \lalchand asyāḥ \nepal \dehlia} mūrterdhyānakaraṇāt\varc{dhyānakaraṇāt \nepal \dehlia \pune}{dhyānakāraṇāt \edprint dhyānāt \lalchand} svargapātālākaśamanuṣyagandharvakinnaraguhyakavidyādharalokasambandhinyaḥ strīyo 'pi \varc{strīyo 'pi \edprint \pune \lalchand}{strīyaḥ \nepal \dehlia}sādhakasya puruṣasya \varc{strīyaḥ sādhakasya puruṣasya \nepal \dehlia}{strīyo 'pi \edprint \pune \lalchand}\notes{sādhakasya puruṣasya}{\englishnote{\small Based on the material at hand it is not possible to determine the better reading at this passage. Due to those doubts I presently lean towards choosing the reading providing more detail: \textit{strīyaḥ sādhakasya puruṣasya} instead of merely \textit{strīyo 'pi}. To complicate the matter a similiar sentence follows only in \nepal and \dehlia, which is left out in \edprint \pune and \lalchand: \textit{pṛthvīloke saṃbaṃdhinyo 'pi striyaḥ vaśyo bhavaṃti} as witnessed in \nepal and \textit{pṛthvīloka saṃbaṃdhinyo pi striyaḥ vasyā bhavaṃti} in \dehlia. This sentence is currently not included into the constituted critical edition due to its clear redundance. However, due to syntactical reasons and to provide a more readable Sanskrit without changing the meaning of the sentence, I tend to produce a conjecture, by adopting the \textit{strīyo 'pi}-reading of \edprint \pune \lalchand and merge it with the \textit{strīyaḥ sādhakasya puruṣasya}-reading of \nepal and \dehlia to arrive at: \textit{strīyo 'pi sādhakasya puruṣasya vaśyā bhavanti}. Please don't freak out.}}vaśyā bhavanti /
%asyā mūrter dhyānakāraṇāt    svargapātālākaśamanuṣyagandharvakinnaraguhyakavidyādharalokasambandhinyaḥ strīyo 'pi vaśyā bhavanti / \edprint
%asyā mūrter dhyānakaraṇāt    svargapātālākāśamanuṣyagandharvakiṃnaraguhyakavidyādharalokasaṃbaṃdhinyaḥ strīyo 'pi vaśyā bhavanti / \pune
%asyā mūrtir dhyānāt          svargapātālākāśamanuṣyagaṃdharvakinnaraguhyakavidyādharalokasambandhinyaḥ strīyo 'pi vaśyā bhavanti /lalchand
%asyāḥ mūrter dhyānakaraṇāt   svargapātāla ākāśamanuṣyagaṃdharvakinnaraguhyakavidyādharalokasaṃbaṃdhinyaḥ strīyaḥ sādhakasya puruṣasya vaśyā bhavanti // \nepal
%asyāḥ mūrter dhyānakaraṇāt   svargapātāla ākāśamanuṣyagaṃdharvakiṃnaraguhyakavidyādharalokasaṃbaṃdhinyaḥ strīyaḥ sādhakasya puruṣasya vaśyā bhavanti // \dehlia
%
% Because of the exercise of meditation on her form the inhabitants of the universe (which are) Humans, Gandharvas, Kinnaras, Guhyakas, Vidyādharas (and their) females, in the heavenly world, underworld and open space are obedient to the will of the practicing person.
%pṛthvīlokasaṃbaṃdhinyo \varc{pṛthvīlokasaṃbaṃdhinyo \dehlia}{pṛthvīloke saṃbaṃdhinyo \nepal} 'pi striyaḥ vaśyā \varc{vaśyā \dehlia}{v%aśyo \nepal} bhavanti / \varc{pṛthvīlokasaṃbaṃdhinyo 'pi striyaḥ vaśyā bhavanti}{\om \edprint \pune \lalchand} 
%The inhabitants of the universe and the earth as well as their females are brought into subjection. 
%
ityatra\varc{ityatra \nepal \pune \lalchand}{ityatra kiṃ \nepal \dehlia} kathyate // 
%ityatra kathyate// /edprint
%ityatra kathyate// \pune
%ityatra kathyate// \lalchand
%ityatra kiṃ kathyate // \nepal
%ityaṃtra kiṃ kathyate // \dehlia 
%which is told in this case.
\endprose
\bigskip
\pstart
\centerline{\begin{english}\textrm{\small{[Description of the fifth Cakra]}}\end{english}}
\pend
\bigskip
\prose
idānīṃ \varc{idānīṃ \nepal \dehlia}{\om \edprint \pune \lalchand}pañcamaṃ kamalaṃ ṣodaśadalaṃ kaṇṭhasthāne \varc{kamalaṃ ṣodaśadalaṃ kaṇṭhasthāne \nepal \dehlia}{kaṇṭasthāne ṣoḍaśadalaṃ kamalaṃ \edprint \pune \lalchand}vartate /
%       pañcamaṃ kaṇṭhasthāne ṣoḍaśadalaṃ kamalaṃ vartate // \edprint
%       paṃcamaṃ kaṃṭhasthāne ṣoḍaśadalaṃ kamalaṃ vartate \pune
%       paṃcamaṃ kaṃṭhasthāne ṣoḍaśadalaṃ kamalaṃ vartate \lalchand
%
%idānīṃ paṃcamaṃ kamalaṃ ṣodaśadalaṃ kaṃṭhasthāne varttate // \nepal
%idānīṃ paṃcamaṃ kamalaṃ ṣodaśadalaṃ kaṃṭhasthāne varttate // \dehlia --------> Was in diesem Falle machen?
%
%Now (follows the description of) the fifth lotus having sixteen petals (which) exists at the location of the throat.
%
%
tanmadhye koṭicandrasamaprabaḥ\varc{koṭicandrasamaprabaḥ \nepal \pune}{koṭicandrasamaprabhā \dehlia \lalchand koṭisūryasamāna \edprint} ekaḥ puruṣo vartate /
%tanmadhye koṭisūryasamāna ekaḥ puruṣo vartate / \edprint
%tanmadhye koṭicaṃdrasamaprabhaḥ ekaḥ puruṣo vartate \pune
%tanmadhye koṭicaṃdrasamaprabhā ekaḥ puruṣo vartate \lalchand
%tanmadhye koṭicaṃdrasamaprabhaḥ ekaḥ puruṣo varttate \nepal
%tanmadhye koṭicaṃdrasamaprabhā ekapuruṣo varttate \dehlia
%In its  middle exists a single person which shines like a thousand moons.
tasya puruṣasya dhyānakaraṇādasādhyarogā\varc{°karaṇāt \nepal \dehlia \pune \lalchand}{°kāraṇāt \edprint} naśyanti /
%tasya puruṣasya dhyānakāraṇādasādhyarogā naśyanti // \edprint
%tasya puruṣasya dhyānakāraṇādasādhyarogā naśyanti // \lalchand
%tasya puruṣasya dhyānakāraṇādasādhyarogā naśyaṃti // \pune
%tasya puruṣasya dhyānakaraṇāt asādhyarogā naśyaṃti // \nepal
%tasya puruṣasya dhyānakaraṇāt / asādhyarogā naśyaṃti // \dehlia
%%Because of the exercise of meditation on this person all diseases which are (otherwise) not possible to be controlled vanish.
ekasahasravarṣaparyaṃtaṃ \varc{°paryaṃtaṃ \edprint \pune \nepal \dehlia}{\om \lalchand} sa \varc{sa \edprint \pune}{\om \lalchand \nepal \dehlia}puruṣo jīvati//
\endprose
\bigskip
\pstart
\centerline{\begin{english}\textrm{\small{[Description of the sixth Cakra]}}\end{english}}
\pend
\bigskip
\prose
īdānīṃ ṣaṣṭhacakraṃ\varc{ṣaṣṭhacakraṃ \nepal \dehlia}{ṣaṣṭhaṃ \edprint \pune ṣaṣṭaḥ \lalchand} ājñānāmakaṃ\varc{ajñānāmakaṃ \nepal \dehlia}{bhrumadhye ājñācakraṃ \edprint \pune \lalchand} vartate /
%Now (there) exists the sixth cakra named \textit{ājñā}. 
%ekasahasravarṣa paryaṃtaṃ sa puruṣo jīvatīdānīṃ ṣaṣṭhaṃ bhrūmadhye ājñācakraṃ vartate// \edprint
%ekasahasravarṣa paryaṃtaṃ sa puruṣo jīvati īdānīṃ ṣaṣṭhaṃ bhrūmadhye ājñācakraṃ vartate// \pune
%ekasahasravarṣa puruṣo jīvati //           īdānīṃ ṣaṣṭhaḥ bhrūmadhye ājñācakraṃ vartate// \lalchand
%ekasahasravarṣa paryaṃtaṃ puruṣo jīvati / idānīṃ ṣaṣṭhacakraṃ ajñānāmakaṃ varttate // \nepal
%ekasahasravarṣa paryaṃtaṃ puruṣo jīvati / idānīṃ ṣaṣṭhacakraṃ ajñānāmakaṃ varttate // \dehlia
%The person lives up to 1001 years. %Now (there) exists the sixth, the \textit{ajñācakra} inbetween the eyebrows.
                                        %dvidalaṃ tanmadhye 'gnijvālākārakamalaṃ kiṃcid vastu vartate / \edprint
                                        %dvidalaṃ tanmadhye agnijvālākārakamalaṃ kiṃcid vastu vartate / \pune
                                        %dvidalaṃ tanmadhye agnijvālākārakamalaṃ kiṃcid vastu vartate / \lalchand
                                                           %agnijvālākārakamalaṃ  kiṃcid vastu vartate / \oxford                                                                           %hier ist mir nicht ganz klar wie ich mich entscheiden soll und ich weiß auch nicht, ob der Apparat so am sinnvollssten ist.  außerdem starer hier auch \oxford ---> muss ich das in einer FN erwähnen?  
%tac cakraṃ bhruvor madhye dvidalakaṃ sthitaṃ // tanmadhye agnijvālākāraṃ akalaṃ kiṃcid vastu varttate / \nepal
%tac cakraṃ bhruvor madhye dvidalakaṃ sthitaṃ // tanmadhye agnijvālākāraṃ akalaṃ kiṃcid vastu varttate / \dehlia                                                   
taccakraṃ bhrūvormadhye dvidalakaṃ sthitaṃ \varc{taccakraṃ bhrūvormadhye dvidalakaṃ sthitaṃ \nepal \dehlia}{dvidalaṃ \edprint \pune \lalchand} / tanmadhye agnijvālākāramakalaṃ\varc{akalaṃ \nepal \dehlia}{\om \edprint \pune \lalchand \oxford}\notes{agnijvālākārakamalaṃ}{\englishnote{\small \oxford starts here. All other folios before are missing.}} kiṃcidvastu vartate /
%In its middle exists a place with a certain two-petalled lotus in the form of blazing fire. \edprint übersetzung
%This two-petalled cakra is situated in the middle of the eyebrows. In its middle exists a certain object (\textit{kiṃcid vastu}) being a form of blazing fire without parts,... \nepal
%na strī pumān / tasya dhyānakāraṇāt puruṣasya    śarīram ajarāmaraṃ bhavati / \edprint
%na strī pumān // tasyā dhyānakaraṇāt puruṣasya    śarīraṃ ajarāmaro bhavati / \oxford
%na strī pumān // tasyā dhyānakaraṇāt puruṣasya    śarīraṃ ajarāmaro bhavati / \lalchand
%na strī na pumān // tasyā dhyānakaraṇāt puruṣasya śarīraṃ ajarāmaro bhavati / \pune
%na strī na pumān / tasya dhyānakaraṇāt puruṣasya śarīram ajarāmaraṃ bhavati \nepal
%na strī na pumān / tasya dhyānakaraṇāt puruṣasya śarīram ajarāmaraṃ bhavati \dehlia
na strī na pumān\varc{na pumān \nepal \dehlia \pune}{pumān \edprint \lalchand \oxford}/ tasya\varc{tasya \edprint \nepal \dehlia}{tasyā \pune \lalchand \oxford} dhyānakaraṇāt\varc{°karaṇāt \nepal \dehlia \pune \lalchand \oxford}{°kāraṇāt \edprint} puruṣasya śarīram ajarāmaraṃ \varc{ajārāmaraṃ \edprint \nepal \dehlia}{ajāramaro \pune \lalchand \oxford} bhavati //
%...not being female not being  male. Because of the exercise of meditation on it the body of the person becomes non-aging and immortal.
\endprose
\bigskip
\pstart
\centerline{\begin{english}\textrm{\small{[Description of the seventh Cakra]}}\end{english}}
\pend
\bigskip
\prose
% idānīṃ saptamaṃ  tālumadhye catuḥṣaṣṭidalaṃ              amṛtapūrṇaṃ vartate / \edprint
% idānīṃ saptamaṃ  tālumadhye catuḥṣaṣṭhidalaṃ             amṛtapūrṇaṃ vartate / \pune
% idānīṃ saptamaṃ  // tāludeśe madhye catuḥṣaṣṭhidala      amṛtapūrṇaṃ vartate / \lalchand
% idānīṃ saptamaṃ  // tāludeśe madhye catuḥṣaṣṭhidala      amṛtapūrṇaṃ vartate / \oxford
% idānīṃ saptamaṃ  cakraṃ     catuḥṣaṣṭhidalaṃ tālumadhye  amṛtapūrṇaṃ varttate // \nepal
% idānīṃ saptamaṃ  cakraṃ     catuḥṣaṣṭhidalaṃ tālumadhye  amṛtapūrṇaṃ varttate // \dehlia
%
%
% Now the seventh cakra having 64 petals and being full of nectar exists in the middle of the palate.
%
%
idānīṃ saptamaṃ cakraṃ\varc{cakraṃ \nepal \dehlia}{\om \edprint \pune \lalchand \oxford} catuḥṣaṣṭhidalaṃ tālumadhye\varc{catuḥṣaṣṭhidalaṃ tālumadhye \nepal \dehlia}{tālumadhye catuḥṣaṣṭhidalaṃ \edprint \pune tāludeśe madhye catuḥṣaṣṭhidala \lalchand \oxford} amṛtapūrṇaṃ varttate /
%
%
%adhikaśobhāyuktam     atiśvetaṃ       tanmadhye       raktavarṇaṃ ghāṃṭikāsaṃjñaikā      karṇikā varttate / \edprint
%adhikataraśobhayuktaṃ atiśvetaṃ       tanmadhye       raktavarṇaṃ ghaṭikāsaṃjñā ekā      karṇikā varttate / \pune
%adhikataraśobhayuktaṃ // atiśvetaṃ // tanmadhye       raktavarṇaṃ ghaṇikāsaṃjñā ekā ekā  karṇikā varttate / \lalchand
%adhikataraśobhayuktaṃ // atiśvetaṃ // tanmadhye       raktavarṇaṃ ghaṃṭikāsaṃjñā ekā ekā karṇikā varttate / \oxford
%adhikataraśobhayuktaṃ atiśvetaṃ       tanmadhye       raktavarṇaṃ ghaṃṭikāsaṃjñā ekā     karṇikā varttate / \nepal
%adhikataraśobhayuktaṃ atiśvetaṃ       tanmadhye       raktavarṇaṃ ghaṃṭikāsaṃjñā ekā     karṇikā varttate / \dehlia
adhikataraśobhayuktaṃ\varc{adhikataraśobhayuktaṃ \nepal \dehlia \pune \lalchand \oxford}{adhikaśobhāyuktam \edprint} atiśvetaṃ tanmadhye raktavarṇaṃ ghaṇṭikāsaṃjñā\varc{ghaṃṭikā° \edprint \oxford \nepal \dehlia}{ghaṭikā° \pune ghaṇikā° \lalchand} ekā\varc{ekā \edprint \pune}{ekā ekā \lalchand \oxford ekāka \nepal \dehlia} karṇikā varttate /
% Being endowed with superabundant beauty, very bright in its middle, red in color exists a single pericarp named ``uvula'' (\textit{ghāṃṭikā}).% 
% tanmadhye bhūmiḥ / \edprint
% tanmadhye bhūmiḥ / \pune
% tanmadhye bhūmiḥ / \lalchand
% tanmadhye bhūmiḥ / \oxford
% tanmadhye bhūmiḥ / \nepal
% tanmadhye bhūmiḥ / \dehlia
%
%In its middle is the place.
%
%
% tanmadhye prakaṭacandrakalā 'mṛtādhārā bhavati           / \edprint
% tanmadhye prakaṭacandrakalā 'mṛtādhārā sravati           / \pune
% tanmadhye prakaṭacandrakalā 'mṛtādhārā sravaṃti          / \lalchand
% tanmadhye prakaṭacandrakalā 'mṛtādhārā sravaṃti          / \oxford
% tanmadhye prakaṭacandrakalā amṛtādhārāsravaṃtī varttate / \nepal
% tanmadhye prakaṭacandrakalā 'mṛtādhārāsravaṃtī varttate / \dehlia %sravantī f. Fluss Nom Sg
tanmadhye bhūmiḥ / tanmadhye prakaṭacandrakalā 'mṛtādhārāsravantī\varc{°sravantī \nepal \dehlia}{sravaṃti \oxford \lalchand sravati \pune bhavati \edprint} vartate\varc{varttate \nepal \dehlia}{\om \edprint \pune \lalchand \oxford} /\notes{amṛtādhārāsravantī vartate}{\englishnote{\small One of the main problems at the current state of constituting the critical edition is the stark deviation in sentence contruction. As mentioned ealier with some exceptions there are two main groups of witnesses. Group a) is \edprint \pune \lalchand \oxford and group b) is \nepal as well as \dehlia. Even though one might assume that there is the possibility of even the former group being closer to the original, this very sentence clearly shows the relation  between the witnesses and suggests a clear direction of the corruption. Here we see how the \textit{vartate} of \nepal and \dehlia was deleted by a scribe in the process of transmission due to misunderstanding the compound \textit{amṛtādhārāsravantī}. The last member of the compound \textit{sravantī} was taken as a verb like \textit{sravaṃti, sravati} or even \textit{bhavati}. This supports the justification for the decisions between variants in cases of deviating sentence construction to usually prefer readings of \nepal and \dehlia.}}
%
%In its middle exists a flow of nectar like a river (\textit{amṛtādhārāsravantī}, appearing from the digit of the moons disc.
%
%
% tasyāḥ kalāyā dhyānakāraṇāt-tasya samīpe maraṇaṃ nāyāti / \edprint -> does not come near to death -> na-ā-yāti
% tasyāḥ kalāyā dhyānakaraṇāttasya samīpe maraṇaṃ nāyāti / \pune
% tasyāḥ karṇikāyā dhyānakaraṇāt tasya samīpe maraṇaṃ na yāti /lalchand
% tasyāḥ karṇikāyā dhyānakaraṇāt tasya samīpe maraṇaṃ na yāti /oxford
% tasyāḥ kalāyāḥ dhyānakaraṇāt tasya samīpe maraṇaṃ nāyāti /nepal
% tasyāḥ kalāyāḥ dhyānakaraṇāt tasya samīpe maraṇaṃ nāyāti /dehlia
%
%Because of the exercise of meditation on this digit death does not come near him. 
%
tasyāḥ kalāyā\varc{kalāyā \edprint \pune}{kalāyāḥ \nepal \dehlia karṇikāyā \lalchand \oxford} dhyānakaraṇāttasya\varc{dhyānakaraṇāt \nepal \dehlia \oxford \lalchand \pune}{dhyānakāraṇāt \edprint} samīpe maraṇaṃ nāyāti\varc{nāyāti \edprint \pune \nepal \dehlia}{na yāti \lalchand \oxford}/%Because of the exercise of meditation on the digit death does not come into proximity of him. 
%
%
%nirantaradhyānād        -amṛtadhārāyāḥ sajīvo bhavati / \edprint
%niraṃtaradhyānātamṛtadhārāplāvanaṃ            bhavati / \pune
%niraṃtaradhyānakaraṇād   amṛtadhārā           sravati / \lalchand
%niraṃtaradhyānakaraṇād   amṛtadhārā           sravati / \oxford
%niraṃtaradhyānakaraṇāt / amṛtadhārā           sravaṃti / \nepal
%niraṃtaradhyānakaraṇāt / amṛtadhārā           sravaṃti / \dehlia
nirantaradhyānakaraṇādamṛtadhārā \varc{°dhyānakaraṇāt \nepal \dehlia \oxford \lalchand}{dhyānāt \edprint \pune}\varc{°dhārā \nepal \dehlia \pune \lalchand \oxford}{°dhārāyāḥ} sravati\varc{sravati \lalchand \oxford}{sravaṃti \nepal \dehlia sajīvo bhavati \edprint plāvanaṃ bhavati \pune} /
%Due to uninterrupted meditation the stream (\textit{dhārā}) of nectar flows. 
% tadā yakṣam-aroga   -pittajvarahṛdayadāha-śiroroga-jihvā--jaḍa-bhāvā  naśyanti / \edprint
% tadā    kṣayaroga   -pittajvarahṛdayadāha-śiroroga-jihvā--jaḍa-bhāvān naśyanti / \pune
% tadā    kṣayaroga   -pittajvarahṛdayadāha     roga-jihvāyājaḍa-bhāvān naśyanti / \lalchand
% tadā    kṣayaroga   -pittajvarahṛdayadāha     roga-jihvāyājaḍa-vān    naśyanti / \oxford
%         kṣayarogaṃ   pittajvarahṛdayadāha-śiroroga jihvāyājaḍa-bhāvā  naśyanti / \nepal %besser kṣayarogaṃ emendieren zu vollem Kompositum? 
%         kṣayaṃ rogaṃ pittajvarahṛdayadāha-śiroroga jihvāyājaḍa-bhāvā  naśyanti / \dehlia
\varc{tadā \edprint \pune \lalchand \oxford}{\om \nepal \dehlia} kṣayarogapittajvarahṛdayadāhaśirorogajihvājaḍabhāvā\varc{kṣayarogaṃ \nepal}{kṣayaroga°\pune \lalchand \oxford kṣayaṃ rogaṃ \dehlia yakṣam aroga \edprint}\varc{°śiroroga- \nepal \dehlia \edprint \pune}{°roga- \lalchand \oxford}\varc{°jihvāyājaḍabhāvā \nepal \dehlia}{jihvāyajaḍabhāvān \lalchand jihvājaḍabhāvān \pune jihvajaḍabhāvā \edprint jihvāyājaḍavān \oxford} naśyanti /
%
%The appearances of emaciation (\textit{kṣayaroga}), fever due to disordered bile (\textit{pittajvara), heartburn (\textit{hṛdayadāha}), head-disease (\textit{śiroroga}) and tongue insensibility (\textit{jihvājaḍa}) vanish. %!!!Krankheiten in Ayurvedabuch checken! medizinische Identifikationen! 
%
%Then the appearances of disease caused by Yakṣas, fever due to disordered bile (\textit}pittajvara), heartburn, head-disease, and tongue insensibility vanishes. \edprint Übersetzung
%
%bhakṣitam  api   viṣan    na bādhate / \edprint
%bhakṣitam  api   viṃṣa    na bādhate / \pune
%bhākṣitam  api   viṣaṃ    na bādhyate / \lalchand
%bhākṣitamār pi   viṣaṃ    na bādhyate / \oxford
%bhakṣitam        viṣamapi na bādhyate / \nepal
%bhakṣitāṃ        viṣamapi na bādhyate / \dehlia
bhakṣitaṃ\varc{bhakṣitam \nepal}{bhakṣitāṃ \dehlia bhakṣitamapi \edprint \pune bhākṣitamapi \lalchand bhākṣitam ār pi \oxford} viṣamapi\varc{viṣamapi \nepal \dehlia}{viṣaṃ \lalchand \oxford viṣan \edprint viṃṣa \pune} na bādhate\varc{bādhate \edprint \pune}{bādhyate \nepal \dehlia \lalchand \oxford} / 
%Also eaten venom doesn't trouble him. 
yadyatra\varc{yadyatra \edprint \pune \nepal}{yadatramapi \lalchand \oxford yadyanna \dehlia} manaḥ sthiraṃ\varc{manaḥ sthiraṃ \edprint \pune}{manasthiram \lalchand \oxford \nepal \dehlia} bhavati //
% yady atra manaḥ sthiraṃ   bhavati / \edprint
% yady atra manaḥ sthiraṃ   bhavati / \pune
% yady atramapi manasthiraṃ bhavati / \lalchand              %VARIANTE UNSICHER!!!WAS MEINT JÜRGEn??
% yady atramapi manasthiraṃ bhavati / \oxford
% yady atra     manasthiraṃ bhavati / \nepal
% yadyanna      manasthiraṃ bhavati / \dehlia
%  If here the mind becomes stable.
\endprose
\bigskip
\pstart
\centerline{\begin{english}\textrm{\small{[Description of the eighth Cakra]}}\end{english}}
\pend
\bigskip
\prose
%idānīṃ brahmarandhrasthāne 'ṣṭamaṃ śatadalaṃ cakraṃ varttate / \edprint
%idānīṃ brahmaraṃdhrasthāne 'ṣṭamaṃ śatadalaṃ cakraṃ vartate / \pune
%idānīṃ brahmaraṃdhrasthāne aṣṭamaṃ śatadalaṃ cakraṃ vartate / \lalchand
%idānīṃ brahmaraṃdhrasthāne aṣṭamaṃ śatadalaṃ cakraṃ vartate / \oxford
%idānīṃ aṣṭamacakraṃ brahmaraṃdhrasthāne śatadalaṃ   vartate / \nepal
%idānīṃ aṣṭamacakraṃ brahmaraṃdhrasthāne śatadalaṃ   vartate / \dehlia
%Now exists the eighth \textit{cakra} having one hundred petals located  at the aperture of Brahman.
idānīṃ aṣṭamacakraṃ brahmarandhrasthāne śatadalaṃ\varc{aṣṭamacakraṃ brahmaraṃdhrasthāne śatadalaṃ \nepal \dehlia}{brahmaraṃdhrasthāne 'ṣṭamaṃ śatadalaṃ cakraṃ \edprint \pune brahmarandhrasthāne aṣṭamaṃ śatadalaṃ cakraṃ \lalchand \oxford} vartate / 
%tasya kamalajātyadharaṇīpīṭha iti saṃjñā / \edprint
%tasya kamalasya jālaṃdharapīṭha iti saṃjñā / \pune
%tasya kamalasya jālaṃdharapīṭha iti saṃjñā ...  \lalchand
%tasya kamalasya jālaṃdharapīṭhasaṃjñā ...  \oxford
%tasya kamalasya jālaṃdharapīṭha iti saṃjñā ...  \nepal
%tasya kamalasya jālaṃdharapīṭha iti saṃjñā ...  \dehlia
tasya kamalasya jālandharapīṭha\varc{kamalasya jālaṃdharapīṭha \pune \lalchand \oxford \nepal \dehlia}{kamalajātyadharaṇīpīṭha \edprint} iti\varc{iti \edprint \pune \lalchand \nepal \dehlia}{\om \oxford} saṃjñā /
%``The (divine) seat of  Jālaṃdhara'' is the designation of the lotus.
%siddhapuruṣasya sthānam / \edprint
%siddhapuruṣasya sthānam / \pune
%siddhapuruṣasya sthānam mūrti vartate // \lalchand                         %%% schwerer Satz -> wie soll ich hier entscheiden?! 
%siddhapuruṣasya sthānam mūrti vartate // \oxford %Zeilensprung
%siddhapuruṣasya sthānam // \nepal
%siddhapuruṣasya sthānam // \dehlia
siddhapuruṣasya sthānam\varc{mūrti vartate \lalchand \oxford}{\om \edprint \pune \nepal \dehlia} /
%
%(It is) the place of the accomplished person.
%
%tanmadhye 'gnidhūmākārarekhā   yādṛśy  ādṛśy ekā  puruṣasya mūrttir varttate / \edprint
%tanmadhye 'gnidhūmākārarekhā   yādṛśī tādṛśy ekā  puruṣasya mūrttir varttate / \pune
%tanmadhye 'gnidhūmākārārekhā  yādṛśī tādṛśy ekā  puruṣasya mūrttir varttate / \lalchand               
%tanmadhye 'gnidhūmākārārekhā  yādṛśī tādṛśy ekā  puruṣasya mūrttir varttate / \oxford
%tanmadhye 'gnidhūmākārāreṣā   yādṛśī tādṛśī ekā  puruṣasya mūrttir varttate / \nepal
%tanmadhye agnidhūmākārāreṣā   yādṛśī tādṛśī ekā  puruṣasya mūrttir varttate / \dehlia
%
tanmadhye 'gnidhūmākārarekhā\varc{'gnidhūmakārarekhā \edprint \pune}{'gnidhūmākārārekhā \lalchand \oxford}{'gnidhūmākārāreṣā \nepal agnidhūmākārāreṣā \dehlia} yādṛśī tādṛśyekā\varc{yādṛśī tādṛśyekā \pune \lalchand \oxford}{yādṛśī tādṛśī ekā \nepal \dehlia yādṛśyādṛśyekā \edprint} puruṣasya mūrtirvartate / 
%
%In its middle exists some single (divine) form of the person (\textit{puruṣa}) being a streak having the form of smoke and fire. 
%
%
%
%tasyā nādir nāṃto 'sti / \edprint
%tasyā nādināṃ 'to sti / \pune
%tasyā nādir nāṃto sti / \lalchand -> vor dem bei allen anderen vorigen Satz!?!?!?! 
%tasyā nādir nāṃto sti / \oxford -> vor dem bei allen anderen vorigen Satz!?!?!?! 
%tasyāḥ nāsthaṃtaḥ ādir api nāsti / \nepal????
%tasyāḥ nāsty aṃtaḥ ādir api nāsti / \dehlia 
%????? unklarer Satz
tasyā\varc{tasyā \edprint \pune \lalchand \oxford}{tasyāḥ \nepal \dehlia} nāstyanta ādirapi nāsti\varc{nāstyaṃtaḥ ādirapi nāsti \dehlia}{tasyāḥ nāsthaṃtaḥ ādirapi nāsti \nepal nādirnāṃto sti \lalchand \oxford tasyā nādirnāṃto 'sti \edprint tasyā nādināṃ 'to sti \pune} /
% Of her exists no end, nor a beginning.

%
%
%-------------------------------
%tasyā  mūrtter dhyānakāraṇāt pratyakṣaṃ niraṃtaraṃ puruṣasyākāśe   gamāgamau   bhavataḥ / \edprint
%tasyā  mūrtter dhyānakaraṇāt pratyakṣaniraṃtaraṃ   puruṣasyākāśe   gamāgamau   bhavataḥ / \pune
%tasyā  mūrtir  dhyānakaraṇāt pratyakṣaniraṃtaraṃ   puruṣasyākāśe   gamāgamau   bhavataḥ / \lalchand         
%tasyā  mūrtir  dhyānakaraṇāt pratyakṣaṃ niraṃtaraṃ puruṣasyākāśe   gamāgamau   bhavataḥ / \oxford
%tasyāḥ mūrttir dhyānakaraṇāt pratyakṣaniraṃtaraṃ   puruṣasya ākāśe gamāgamau   bhavataḥ / \nepal
%tasyāḥ mūrtir  dhyānakaraṇāt pratyakṣaniraṃtaraṃ   puruṣasya ākāśe gamāgamau   bhavataḥ / \dehlia
tasyā\varc{tasyā \edprint \pune \lalchand \oxford}{tasyāḥ \nepal \dehlia} mūrterdhyānakaraṇāt\varc{mūrter\edprint \pune}{mūrtir \lalchand \oxford \dehlia mūrttir \nepal}\notes{\conj pratyakṣaṃ nirantaraṃ}{\englishnote{\small Even though every single witness at hand transmits the latter reading right after \textit{°karaṇāt}, several considerations make it reasonable to conject that the original sentence is corrupted and was written without it. The main consideration to assume the corruption is that \textit{pratyakṣaṃ nirantaraṃ} is ungrammatical. The second is that the sentence is way more meaningful without it. The third that two sentences later we get the phrase in a meaningful context. Due to the last consideration my best guess is an interlace at an early stage of transmission.}}\varc{\conj. °karaṇāt pratyakṣaṃ nirantaraṃ \edprint \oxford}{°karaṇāt pratyakṣanirantaraṃ \nepal \dehlia \pune \lalchand} puruṣasyākāśe gamāgamau bhavataḥ /
%BEDEUTUNG DES SATZES BIS JETZT UNKLAR! Idee: Zeilensprung aus übernächstem Satz! Streiche pratyakṣaṃ niraṃtaraṃ und der Satz ergibt Sinn!  
%gamāgamau nom.  dual = coming and going ; bhavataḥ = 3p du ind pres von bhū
%Due to the exercise of meditation on this (divine) form both coming and going of the person in space occurs. 
%Kolloquium: Meinung zu Kompositum pratyakṣaniraṃtaraṃ = macht wenig Sinn oder?
%-------------------------------
%pṛthvīmadhye  sthitasyāpi pṛthvībādho          na bhavati / \edprint
%pṛthvīmadhye  sthitasyāpi    pṛthaka              bhavati   \pune %Zeilenspringer führt zu Verlust von Zeile in Pune
%pṛthvīmadhye  sthitasyāpi pṛthvībādho          na bhavati / \lalchand
%pṛthivīmadhye sthitasyāpi // pṛtvībādho        na bhavati // /oxford
%pṛthvīmadhye  sthitāvapi  pṛthvī kṣa?to bādho  na bhavati // \nepal
%pṛthvīmadhye  sthitāvapi  pṛthvīkṣato bādho    na bhavati // \dehlia
%
pṛthvīmadhye\varc{pṛtvīmadhye \edprint \nepal \dehlia \pune \lalchand}{pṛthivīmadhye \oxford} sthitasyāpi\varc{sthitasyāpi \edprint \lalchand \oxford \pune}{sthitāvapi \nepal \dehlia} pṛthvībādho\varc{pṛthvībadho \edprint \lalchand \oxford}{pṛtvīkṣato bādho \nepal \dehlia pṛtahaka \pune} na\varc{na \edprint \lalchand \oxford \nepal \dehlia}{\om \pune} bhavati / 
%Affliction from the earth-element does not arise (anymore) even if he is situated in the middle of the earth.  
%------------------------------
%sakalān pratyakṣaṃ niraṃtaraṃ paśyati ca pṛthagbhavati / \edprint
%sakalāḥ pratyakṣaṃ niraṃtara paśyatī  ca pṛthak bhavati // \oxford
%sakalāḥ pratyakṣaṃ niraṃtara paśyatī  ca pṛthak bhavati / \lalchand
%sakalāpratyakṣaniraṃtaraṃ    paśyati  ca pṛthak ca bhavati // \nepal
%sakalāpratyakṣaniraṃtaraṃ paśyati ca pṛthak pṛthak bhavati \dehlia
% omitted completely in \pune
%
sakalaṃ pratyakṣaṃ nirantaraṃ\varc{sakalāpratyakṣaṃ nirantaraṃ}{\emend sakalāpratyakṣaniraṃtaram \nepal \dehlia sakalāḥ pratyakṣaṃ niraṃtara \oxford \lalchand sakalān pratyakṣaṃ niraṃtaraṃ \edprint \om \pune} paśyati\varc{paśyati \edprint \nepal \dehlia}{paśyatī \oxford \lalchand \om \pune} ca pṛthagbhavati\varc{pṛthagbhavati \edprint}{pṛthak bhavati \oxford \lalchand pṛthak ca bhavati \nepal pṛthak pṛthak bhavati \dehlia \om \pune} /
%He constantly sees everything in front of his eyes and he becomes separated (from the material world).
%
%------------------------------
%
%atiśayenāyur vardhate / \edprint 
%atīśayanāyur vardhayate / \oxford
%atīśayanāyur vardhayate // \lalchand
%atiśayena āyur varddhate // \nepal
%atiśayena āyur varddhate // \dehlia
%atiśayenāyur vardhate \pune
%
%
% The force of life increases eminently. 
%
atiśayenāyurvardhate\varc{atiśayenāyur \edprint \pune}{atīśayanāyur \oxford \lalchand atiśayena āyur \nepal \dehlia} //
%
%
%------------------------------
%
\endprose
\bigskip
\pstart
\centerline{\begin{english}\textrm{\small{[Description of the ninth Cakra]}}\end{english}}
\pend
\bigskip
\prose
%
%idānīṃ navamacakrasya   bhedāḥ kathyante / \edprint
%idānīṃ navamacakrasya   bhedāḥ kathyante / \pune
%idānīṃ navamaṃ cakrasya bhedāḥ kathyate // \oxford
%idānīṃ navamacakrasya   bhedāḥ kathyate \lalchand
%idānīṃ navamacakrasya   bhedāḥ kathyaṃte // \nepal
%idānīṃ navamacakrasya   bhedāḥ kathyaṃte // \dehlia
%
%Now the divisions/differentiations of the ninth cakra are explained.
%
idānīṃ navamacakrasya\varc{navamacakrasya \edprint \pune \lalchand \nepal \dehlia}{navamaṃ cakrasya} bhedāḥ kathyante\varc{kathyante \edprint \pune \nepal \dehlia}{kathyate \lalchand \oxford}/
% !!!Ist es nicht seltsam, dass wir hier bhedāḥ haben?!!! Welche Divisions? Mal weiterlesen. Vielleicht wirds dann klarer. 
%
%------------------------------
%
%tasya mahāśūnyacakram    iti  saṃjñā / \edprint
%tasya mahāśūnyacakram    iti  saṃjñā / \pune
%tasye mahāśūnye cakram   iti  saṃjñā \oxford
%tasya mahāśūnye cakram   iti  saṃjñā \lalchand
%tasya mahāśūnye cakreti       saṃjñā // \nepal
%tasya mahāśūnyacakreti        saṃjñā // \dehlia
%
%
tasya mahāśūnyacakramiti\varc{mahāśūnyacakramiti \edprint \pune}{mahāśūnye cakram iti \oxford \lalchand mahāśūnye cakreti \nepal mahāśūnyacakreti \dehlia} saṃjñā /
%
%The designation of it is ``the \textit{cakra} of the great void (\textit{mahāśūnyacakra})''.
%
%------------------------------
%
%tadupary aparaṃ kimapi nāsti / \edprint
%tadupary aparaṃ kimapi nāsti \pune
%tadupary        kimapi nāsti \oxford ??-> auch mögliche Lesart
%tadupari        kimapi nāsti \lalchand
%tadupari aparaṃ kiṃapi nāsti / \nepal
%tadupari aparaṃ kiṃapi nāsti / \dehlia
%
%kim api: somewhat, to a considerable extent, rather, much more, still, further. Śa
%
%Above that there is no other. 
%
taduparyaparaṃ\varc{tadupary \edprint \pune}{tadupari \lalchand \nepal \dehlia}\varc{aparaṃ \edprint \pune \nepal \dehlia}{\om \lalchand \oxford} kimapi nāsti / 
%
%------------------------------
%
%tadeva mahāsiddhacakraṃ kathyate // \edprint
%tadeva mahāsiddhacakraṃ kathyate    \pune 
%tadeva mahāsiddhacakraṃ kathyate // \oxford
%tadeva mahāsiddhacakraṃ kathyate // \lalchand
%tadeva mahāsiddhacakraṃ kathyate // \nepal
%tadeva mahāsiddhacakraṃ kathyate // \dehlia
%
%Therefore it is declared to be the \textit{cakra} of the great perfection (\textit{mahāsiddhacakra}).
%
tadeva mahāsiddhacakraṃ kathyate /
%------------------------------
%
%tasya pūrṇagiripīṭha              etadṛśaṃ nāma /\edprint
%tasya pūrṇagiripīṭham iti         etādṛśaṃ nāma \pune
%tasya pūrṇagiripīṭham iti saṃjñā  etādṛsaṃ nāma \oxford
%tasya pūrṇagiripīṭham iti saṃjñā  etādṛsaṃ nāma \lalchand
%tasya cakrasya  pūrṇagiri         etādṛśaṃ nāma / \nepal
%tasya cakrasya  pūrṇagiri         etādṛśaṃ nāma / \dehlia
%
tasya\varc{tasya \edprint \pune \oxford \lalchand}{tasya cakrasya \nepal \dehlia} pūrṇagiripīṭhamiti\varc{pūrṇagiripīṭhamiti \pune \oxford \lalchand}{pūrṇagiripīṭha \edprint pūrṇagiri \nepal \dehlia}\varc{saṃjñā \oxford \lalchand}{\om \edprint \pune \nepal \lalchand} etādṛśaṃ\varc{etādṛśaṃ \edprint \pune}{etādṛsaṃ \lalchand \oxford} nāma /
%!!!Varianten im Kolloquium diskutieren?!
%
%A(nother) name of it is ``(divine) seat of Pūrṇagiri''.   
%
%
%------------------------------
%
%tasya mahāśūnyacakrasya madhye ūrdhvamukham iti raktavarṇaṃ sakalaśobhāspadam \edprint
%tasya mahāśūnyacakrasya madhye ūrdhvamukham iti raktavarṇa-sakalaśobhāspadaṃ \pune
%tasya mahāśūnyacakrasya madhye ūrdhvamukhem iti raktavarṇaṃ sakalaśobhāspadaṃ // \oxford    
%tasya mahāśūnyacakrasya madhye ūrdhvamukham iti raktavarṇaṃ sakalaśobhāspadaṃ // \lalchand
%tasya mahāśūnyacakramadhye     ūrdhvamukhaṃ atiraktavarṇaṃ sakalaśobhāspadaṃ / \nepal
%tasya mahāśūnyacakramadhye     ūrdhvamukhaṃ atiraktavarṇaṃ sakalaśobhāspadaṃ / \dehlia
%
%
%anekakalyāṇapūrṇaṃ sahasradalan   ekaṃ kamalaṃ  varttate / \edprint
%anekakalyāṇapūrṇaṃ sahasradalaṃ   ekaṃ kamalaṃ  vartate \pune
%anekakalyāṇapūrṇa--sahasradalaṃ   ekaṃ kamalaṃ  vartato \oxford
%anekakalyāṇapūrṇaṃ sahasradalaṃ   ekaṃ kamalaṃ  vartate \lalchand
%anekakalyāṇapūrṇaṃ sahasradalaṃ   ekakamalaṃ    varttate \dehlia
%anekakalyāṇapūrṇaṃ sahasradalaṃ   ekaṃ kamalaṃ  vartate \nepal
%Fragezeichen in |nepal ... schreiber Einfügung? 
%
%
tasya mahāśūnyacakramadhye\varc{mahāśūnyacakramadhye \nepal \dehlia}{mahāśūnyacakrasya madhye \edprint \pune \lalchand \oxford} ūrdhvamukhaṃ\varc{ūrdhvamukhaṃ \nepal \dehlia}{urdhvamukham \edprint \pune \lalchand urdhamukhem \oxford} atiraktavarṇaṃ\varc{atiraktavarṇaṃ \nepal \dehlia}{iti raktavarṇaṃ \edprint \lalchand \oxford iti raktavarṇa° \pune} sakalaśobhāspadaṃ /
%
%
anekakalyāṇapūrṇaṃ\varc{anekakalyāṇapūrṇaṃ \edprint \pune \lalchand \dehlia \nepal}{anekakalyāṇapūrṇa° \pune} sahasradalaṃ ekaṃ kamalaṃ\varc{ekaṃ kamalaṃ \edprint \pune \lalchand \oxford \nepal ekakamalaṃ \dehlia} vartate\varc{vartate \edprint \pune \lalchand \nepal \dehlia}{vartato \oxford} /
%
%In the middle of the \textit{mahāśūnyacakra} exists one lotus facing upward, very red in color with a thousand petals - an abode of brilliance and wholeness.
%------------------------------
%yasya parimalo manaso vacaso na gocaraḥ // \edprint
%yasya parimalo manasā vacasā gocaraḥ / \lalchand
%yasya parimalo manasā vacasā na gocaraḥ / \oxford
%yasya parimalo manasā vacasā na gocaraḥ / \nepal
%yasya parimalo manasā vacasā na gocaraḥ / \dehlia
%yasya parimalo manasā vacasā na gocaraḥ / \pune
%
%
yasya parimalo manasā\varc{manasā \pune \lalchand \oxford \nepal \dehlia}{manaso \edprint} vacasā\varc{vacasā \pune \lalchand \oxford \nepal \dehlia}{vacaso \edprint}\notes{manasā vacasā}{\englishnote{\small All manuscripts at hand share this usage of the instrumentals. Only the printed edition conjectures the forms into the exspected genitiv.}}na\varc{na \edprint \pune \oxford \nepal \dehlia}{\om \lalchand} gocaraḥ /
%Whose fragrance is not in range of mind and speech. 
%Dessen Duft ist nicht in Reichweite von Geist und Sprache. 
%
%------------------------------
%
%tasya kamalasya madhye trikoṇarūpa-ikā karṇikā varttate / \edprint
%tasya kamala----madhye     trikoṇārūpā ekā karṇikā varttate/ \pune
%tasya kamalasya madhye trikoṇarūpā ekā karṇikā varttate/ \lalchand
%tasya kamalasya madhye trikoṇarūpā ekā karṇikā varttate/ \oxford
%tasya kamalasya madhye trikoṇarūpā eka karṇikā varttate/ \nepal
%tasya kamalasya madhye trikoṇarūpā ekā karṇikā varttate/ \dehlia
%
%
tasya kamalasya\varc{kamalasya \edprint \lalchand \oxford \nepal \dehlia}{kamala° \pune} madhye trikoṇarūpaikā\varc{trikoṇarūpaikā \edprint}{trikoṇarūpā ekā \pune \lalchand \oxford \dehlia trikoṇarūpā eka \nepal} karṇikā vartate /
%
%In the middle of this lotus exists one pericarp having the shape of a triangle. 
%------------------------------
%
%tatkarṇikāmadhye saptadaśī         niraṃjanarūpā kalā varttate/ \edprint
%tatkarṇikāmadhye saptadaśireṇa ekā niraṃjanarūpā kalā vartate// \lalchand
%tatkarṇikāmadhye saptadaśireṇa ekā niraṃjanarūpā kalā vartate// \oxford
%tatkarṇikāmadhye saptadaśī     ekā niraṃjanarūpā kalā vartate// \pune
%tatkarṇikāmadhye saptadaśī     ekā niraṃjanarūpā kalā vartate// \nepal
%tatkarṇikāmadhye saptadaśī     ekā niraṃjanarūpā kalā vartate// \dehlia
%
%
tatkarṇikāmadhye saptadaśī\varc{saptadaśī \edprint \pune \nepal \dehlia}{saptadaśireṇa \lalchand \oxford} ekā\varc{ekā \pune \nepal \dehlia \lalchand \oxford}{\om \edprint} nirañjanarūpā kalā vartate /\notes{saptadaśī}{\englishnote{\small A \textit{saptadaśī kalā} appears frequently in Śaiva literature. References need to be added here.}}
%
%In the middle of the pericarp exists one seventeenth digit in the shape of a immaculé form. 
%
%------------------------------
%
%koṭisūryasamaprabhaṃ kalāyās tejo vartate / \edprint
%koṭisūryasamaprabhā kalāyās tejo vartate / \lalchand
%koṭisūryasamaprabhā kalāyās tejo vartate / \oxford
%koṭisūryasamaprabha kalāyās tejo vartate / \pune
%koṭisūryasamaprabhaṃ kalāyās tejo vartate / \nepal
%koṭisūryasamaprabhaṃ kalāyās tejo vartate / \dehlia
%
%
koṭisūryasamaprabhaṃ\varc{°samaprabhaṃ \edprint \nepal \dehlia}{°samaprabhā \lalchand \oxford °samaprabhā \pune} kalāyāstejo vartate /
%
%There exists a light of the part shining like a thousand suns. 
%------------------------------
%
%param udbhavo nāsti / \edprint
%parim uṣṇabhavo nāsti / \pune
%parim uṣṇabhavo nāsti / \lalchand
%parim uṣṇabhavo nāsti / \oxford
%parim uṣṇabhāvo nāsti / \nepal
%parim auṣṇabhāvo nāsti / \dehlia
%
%
paramuṣṇabhavo\varc{param° \edprint}{parim° \pune \oxford \nepal \dehlia} nāsti / 
%
%(But) excessive heat is not arising. 
%------------------------------
%
%koṭicandrasamaprabhā śītalaṃ paraṃ śītabhāvo nāsti / \edprint
%koṭicandrasamaprabhā śītalaṃ paraṃ śītabhavo nāsti / \pune
%koṭicandrasamaprabhā śītalaṃ paraṃ śītabhavo nāsti / \NICHT IN LALCHAND!!!!
%koṭicandrasamaprabhā śītalaṃ paraṃ śītabhavo nāsti / \oxford
%
%Shining like a thousand moons, although being cool, cold is not arising.
%
%koṭicandrasamaprabhaṃ śītalaparaṃ bhavo nāsti / \nepal
%koṭicaṃdrasamaprabhaṃ śītalaparaṃ bhavo nāsti / \dehlia
%
%
koṭicandrasamaprabhaṃ\varc{koṭicandrasamaprabhaṃ}{koṭicandrasamaprabhā \edprint \pune \oxford \om \lalchand} śītalaparaṃ\varc{śītalaparaṃ \nepal \dehlia}{śītalaṃ paraṃ \edprint \pune \oxford \om \lalchand} bhavo\varc{\nepal \dehlia śītabhavo \edprint \pune \oxford \om \lalchand} nāsti /
%
%Shining like a thousand moons, excess of cold is not arising.
%
%------------------------------
%
%asyāḥ kalāyā  dhyānayogāt    sādhakasya manasi duḥkhaṃ na bhavati / \edprint
%asyāḥ kalādhyānayogāt        sādhakasya manasi duḥkhaṃ na bhavati / \pune

%asyāḥ kalāyāḥ  dhyānakaraṇāt sādhakasya manasi duḥkhaṃ na bhavati / nepal
%asyāḥ kalāyāḥ dhyānakaraṇāt  sādhakasya manasi duḥkhaṃ na bhavati / dehlia
%
%asyāḥ kalāyā dhyānayogāt     sādhakasya manasi duḥkhaṃ bhavati /oxford
%asyāḥ kalāyā dhyānayogāt     sādhakasya manasi duḥkhaṃ bhavati /lalchand
%
%
asyāḥ kalāyā dhyānakaraṇāt\varc{\emend kalāyāḥ dhyānakaraṇāt \nepal \dehlia}{kalāyā dhyānayogāt \nepal \dehlia kalādhyānayogāt \pune} sādhakasya manasi duḥkhaṃ na\varc{na \edprint \pune \nepal \dehlia}{\om \oxford \lalchand} bhavati /
%
%Due to the exercise of meditation upon the digit suffering does not arise in the mind of the practitioner (anymore). 
%------------------------------
%
%tadupari anaṃtaparamānandasya sthānam / \edprint
%tadupari anaṃtaparamānandasya sthānaṃ  \pune
%tadupari anantaparamānaṃdasya sthānam / \nepal
%tadupari anantaparamānaṃdasya sthānaṃ / \dehlia
%tadupari anantaparamānaṃdasya sthānam vartate/ \oxford
%tadupari anaṃtaparamānaṃdasya sthānam vartate/ \lalchand
%
%
tadupari anantaparamānaṃdasya sthānam\varc{sthānam \edprint \pune \nepal \dehlia}{sthānam vartate \oxford \lalchand}/
%
%Above that is the place of infinite supreme bliss.
%
%
%
%------------------------------
%
%tatrordhvaśaktiḥ / \edprint
%tatordhvaśaktiḥ \pune
%tatrordhvaśaktiḥ / \nepal
%tatra ūrdhva śaktiḥ / \nepal
%rdhaśakti ardhaśakti \oxford
%rdhaśakti ardhaśakti \lalchand
%
%
%
tatrordhvaśaktiḥ\varc{tatrordhvaśaktiḥ \edprint \nepal}{tatordhvaśaktiḥ \pune tatra ūrdhva śaktiḥ \dehlia rdhaśakti ardhaśakti \lalchand \oxford}/
%
%There above is \textit{śakti},
%------------------------------
%
%etādṛśī  saṃjñā ekā kalā vartate / \edprint
%ekādaśā  saṃjñā ekā kalā vartate \pune
%etādṛśī  saṃjñā ekā kalā vartate / \nepal
%etādṛsaṃ saṃjñā ekā kalā vartate / \dehlia
%ekādaśā  saṃjñā ekā kalā vartate / \oxford
%ekādaśā  saṃjñā ekā kalā vartate / \lalchand
%
%
etādṛśī\varc{etādṛśī \edprint \nepal}{etādṛśaṃ \dehlia ekādaśā \pune \lalchand \oxford} saṃjñā ekā kalā vartate / 
%
%Being designated as such she is one single digit. 
%
%------------------------------
%
%asyāḥ kalāyā dhyānakāraṇāt  puruṣo yadicchati / \edprint
%asyāḥ kalāyā dhyānakāraṇāt  puruṣo yadicchati ?Zeichen? \pune
%asyāḥ kalāyā dhyānakāraṇāt  puruṣo yadicchati  tad bhavati \nepal
%asyāḥ kalāyā dhyānakāraṇā   puruṣo yadicchati  tad bhavati \dehlia
%asyāḥ kalāyā dhyānakāraṇāt / puruṣo yadicchati / \oxford
%asyāḥ kalāyā dhyānakāraṇāt / puruṣo yadicchati / \lalchand
%
%
asyāḥ kalāyā dhyānakāraṇāt\varc{dhyānakāraṇāt \edprint \pune \oxford \lalchand \nepal}{dhyānakaraṇā \dehlia} puruṣo yadicchati tadbhavati\varc{tadbhavati \nepal \dehlia}{\om \edprint \pune \lalchand \oxford} / 
%
%Due to the exercise of meditation on this part the person manifests whatever he wishes for.
%
%------------------------------
%
%tasya sukhabhogavataḥ / \edprint
%tasya sukhabhogavataḥ \pune
%rājyasukhabhogavataḥ \nepal
%rājyasukhabhogavṛtaḥ \dehlia !!!
%tasya khaṃ bhogavataṃ / \oxford
%tasya sukhaṃ bhogavaṃtaṃ / \lalchand
%
rājyasukhabhogavṛtaḥ\varc{rājyasukhabhogavṛtaḥ \dehlia}{rājyasukhabhogavataḥ \nepal tasya sukhabhogavataḥ \edprint \pune tasya sukhaṃ bhogavaṃtaṃ \lalchand tasya khaṃ bhogavataṃ} /
%
%He is furnished with royal pleasure and enjoyment. 
%
%------------------------------
%
%strīmadhye vilāsavataḥ    saṃgītavilāsavataḥ vinodaprekṣāvataḥ      puruṣasya pratidinaṃ śuklapakṣe candrakalāvat   kalā    vardhate/ \edprint
%strīmadhye vilāsavataḥ    saṃgītavinodaprekṣāvataḥ eva              puruṣasya pratidinaṃ śuklapakṣe candrakalāvat   kalā    vardhate / \pune
%strīmadhye vilāsavaṃtaṃ   saṃgītaṃ prekṣāvatāḥ // evaṃ              puruṣasya pratidinaṃ śuklapakṣe caṃdrakalāvat / kalā    vartate / \lalchand
%strīmadhye vilāsavaṃtaṃ   saṃgītaṃ vinodavaṃtaṃ prekṣāvaṃtāḥ // eva puruṣasya pratidinaṃ śuklapakṣe caṃdrakalāvat / kalā    vartate / \oxford
%strīmadhye vilāsavataḥ    saṃgītavinodaprekṣyāvataḥ    evaṃ         puruṣasya pratidinaṃ śuklapakṣe candrakalā vṛddhivato?   vardhate / \nepal
%strīmadhye vilāsavataḥ // saṃgītavinodaprekṣyāvataḥ // evaṃ         puruṣasya pratidinaṃ śuklapakṣe candrakalā vṛddhivato    vardhate / \dehlia
%
strīmadhye vilāsavataḥ\varc{vilāsavataḥ \edprint \pune \nepal \dehlia}{vilāsavaṃtaṃ \lalchand \oxford} saṃgītavinodaprekṣyāvataḥ\varc{saṃgītavinodaprekṣyāvataḥ \nepal \dehlia}{saṃgītavilāsavataḥ vinodaprekṣāvataḥ \edprint saṃgītavinodaprekṣāvataḥ \pune saṃgītaṃ prekṣāvatāḥ \lalchand saṃgītaṃ vinodavaṃtaṃ prekṣāvaṃtāḥ \oxford} eva\varc{eva \oxford \pune}{evaṃ \nepal \dehlia \lalchand \om \edprint} puruṣasya pratidinaṃ śuklapakṣe candrakalāvat kalā\varc{candrakalāvat kalā \edprint \pune \lalchand \oxford}{candrakalā vṛddhivato \nepal \dehlia} vardhate\varc{vardhate \edprint \pune \nepal \dehlia}{vartate \lalchand \oxford} /
%(Selbst) bei einem Menschen, der sich inmitten von Frauen vergnügt, (und) ein Musikvergnügen
%ansieht, wächst täglich die Kraft (kalā = śakti?) wie die "kalā" (Phase) des Mondes in der hellen Monatshälfte.
%The \textit{kalā} of a person grows daily, like the \textit{kalā} of the moon in the bright half of the month, even amusing oneself amongst women and watching a musical pleasure.
%(Even) amusing oneself amongst women, and watching musical pleasures, the \textit{kāla} of the person grows daily like the \textit{kalā} of the moon in the bright half of the month. 
%------------------------------
%
%puṇyapāpe 'sya śarīraṃ na spṛśataḥ / \edprint
%puṇyapāpe asya śarīrena spṛśataḥ / \nepal
%puṇyapāpe asya śarīrena spṛśataḥ / \dehlia
%puṇyapāpe asya śarīrasya na spṛśataḥ // \oxford
%puṇyapāpe asya śarīrasya na spṛśataḥ // \lalchand
%omitted in \pune
%
puṇyapāpe\varc{puṇyapāpe \edprint \lalchand \oxford \nepal \dehlia}{\om \pune} 'sya\varc{'sya \edprint}{asya \nepal \dehlia \oxford \lalchand \om \pune} śarīrasya\varc{śarīrasya \lalchand \oxford}{śarīraṃ \edprint śarīrena \nepal \dehlia \om \pune} na\varc{na \edprint \oxford \lalchand}{\om \nepal \dehlia \pune} spṛśataḥ\varc{spṛśataḥ \edprint \lalchand \oxford \nepal \dehlia}{\om \pune} /
%
%His body is not affected by merit and sin. 
%------------------------------
%
%nirantaradhyānakaraṇāt     nijasvarūpaṃ prakāśanasāmarthyaṃ bhavati / \edprint
%nirantaradhyānakaraṇāt /   nijasvarūpaprakāśasāmarthyaṃ bhavati / \nepal <-----
%nirantaradhyānakaraṇāt /   nijasvarūpaprakāśasāmarthyaṃ bhavati / \dehlia
%niraṃtaraṃ dhyānakaraṇāt   nijasvarūpaprakāśasāmarthyaṃ bhavati / \oxford
%niraṃtaraṃ dhyānakaraṇāt// nijasvarūpaprakāśasāmarthyaṃ bhavati / \lalchand
%omitted until .....        nijasvarūpaprakāśasāmarthyaṃ bhavati / \pune
%
nirantaradhyānakaraṇāt\varc{nirantaradhyānakaraṇāt \edprint \nepal \dehlia}{niraṃtaraṃ dhyānakaraṇāt \oxford \lalchand \om \pune} nijasvarūpaprakāśasāmarthyaṃ\varc{nijasvarūpaprakāśasāmarthyaṃ \lalchand \oxford \nepal \dehlia \pune}{nijasvarūpaṃ prakāśanasāmarthyaṃ \edprint} bhavati /
%
%Due to uninterrupted meditation the power of the light of the innate nature arises. 
%------------------------------
%
%dūrasthopi ca dūrasthavastu samīpa iva  paśyati // \edprint
%dūrasthamapi                samīpamiva  paśyati // \nepal
%dūrasthamapy-arthaṃ         samīpa iva  paśyati // \dehlia
%dūrasthamapi padārthaṃ      samīpa iva  paśyati // \oxford
%dūrasthamapi parārthaṃ      samīpa iva  paśyati // \lalchand
%dūrasthamapi padārthaṃ      samīpa iva  paśyati // \pune
%------------------------------
%
dūrasthamapyarthaṃ\varc{dūrasthamapyarthaṃ \dehlia}{dūrasthamapi padārthaṃ \oxford \pune durasthamapi parārthaṃ \lalchand sūrastamapi \nepal ca dūrasthavastu \edprint} samīpa\varc{samīpa \dehlia \edprint \lalchand \oxford \pune}{samīpam \nepal} iva paśyati //
%
%He sees remotely located objects as if they'd be near.
%
%
%
%
%
%
%This is a test for Github! 
%
%
%
%
%
%
%
\endprose
\bigskip
\pstart
\centerline{\begin{english}\textrm{\small{[Lakṣyayoga, the yoga of fixation]}}\end{english}}
\pend
\bigskip
\prose
%------------------------------
%
%idānīṃ sukhasādhyo lakṣyayogaḥ kathyate / \edprint
%idānīṃ sukhasādho  lakṣyayogaḥ kathyate / \pune 
%idānīṃ sukhasādhyo lakṣyayohaḥ kathyate / \nepal
%idānīṃ sukhasādhyo lakṣyayohaḥ kathyate / \dehlia
%
%Now the yoga of fixation{\textit{lakṣyayoga}}, which is easily accomplished is explained. 
%------------------------------
%asya lakṣyayogasya paṃca bhedā bhavanti   ūrdhvalakṣyam / adholakṣyam / lakṣyam / bāhyalakṣyam / aṃtaralakṣyam / \edprint
%asya lakṣyayogasya paṃca bhedā bhavanti   ūrdhvalakṣyam  adholakṣyam / madhyalakṣyam  bāhyalakṣyam  aṃtaralakṣyam / \pune
%     lakṣyayogasya paṃcabhedā  bhavaṃti //urdhvalakṣya adholakṣya bāhyalakṣya madhyalakṣya antaralakṣya // \nepal
%     lakṣyayogasya paṃcabhedā  bhavaṃti //urdhvalakṣya adholakṣya bāhyalakṣya madhyalakṣya antaralakṣya // \unbeaknnt
%
%Of this yoga of fixation (\textit{lakṣyayoga}) there are five subdivisions: 1. The upward directed fixation {\textit{ūrdhvalakṣya}), 2. the downward directed fixation (\textit{adholakṣya}),3. the central fixation (\textit{madhyalakṣya}) 4. the outer fixation (\textit{baḥyalakṣya}), 5. the inner fixation (\textit{antaralakṣya}).
%------------------------------
%prathamam ūrdhvalakṣyaṃ kathyate / \edprint
%prathamam ūrdhvalakṣyaḥ kathyate / \pune
%prathamaṃ urdhvalakṣaḥ kathyate / \nepal
%prathamaṃ urdhvalakṣaḥ kathyate / \dehlia
%
%
%At first the upward directed fixation{\textit{adholakṣya} is explained. 
%------------------------------
%
%ākāśamadhye dṛṣṭiḥ / \edprint
% \om                 \pune
%ākāśamadhye dṛṣṭiḥ / \nepal
%ākāśamadhye dṛṣṭiḥ / \dehlia
%
%
%The gaze (\textit{dṛṣṭi)) [should be] in the middle of the sky. 
%------------------------------
%
%kadā ca mana ūrdhvaṃ kṛtvā sthāpayati / \edprint
%atha ca mana ūrdhvaṃ kṛtvā sthāpyate / \pune
%atha ca // mana urdhvaṃ kṛtvā sthāpyate / \nepal
%atha vā mana ūrdhaṃ kṛtvā sthāpyate \dehlia
%
%When having caused the mind to be directed upwards, it is caused to be fixed there. 
%------------------------------
%
%etasya lakṣyasya dṛḍhakaraṇāt   parameśvarasya tejasā saha dṛṣṭeraikyaṃ  bhavati / \edprint
%etasya lakṣyasya dṛḍhakaraṇāt   parameśvarasya tejasā saha dṛṣṭeraikyaṃ  bhavati / \pune
%etasya lakṣyasya dṛḍhīkaraṇāt / parameśvarasya tejasā saha dṛṣteḥ aikyaṃ bhavati / \nepal
%etasya lakṣasya dṛḍhīkaraṇāt // parameśvarasya tejasā saha dṛṣṭeḥ aikyaṃ bhavati // \dehlia
%
%Due to the exercise of stabilizing of this fixation (\textit{lakṣya}) arises unity of the gazing point (\textit{dṛṣṭi}) with the light of the highest lord (\textit{parameśvara}). 
%
%------------------------------
%atha cākāśamadhye yaḥ kaścid adṛṣṭaḥ padārtho bhavati / \edprint
%atha cākāśamadhye yaḥ kaścid adṛṣṭaḥ padārtho bhavati / \pune
%atha ca ākāśamadhye yaḥ kaścit adṛṣtaḥ padārthe bhavati / \nepal
%atha ca ākāśamadhye yaḥ kaścit adṛsṭaḥ padārtho bhavati / \dehlia
%
%And then an indefinable invisible object arises in the middle of the sky.
%
%------------------------------
%
%sa sādhakasya dṛṣṭigocaro bhavati// \edprint
%sa sādhakasya dṛṣṭigocaro bhavati// \pune
%sādhakasya dṛṣṭigocaro bhavati// \lalchand
%sa sādhakasya dṛṣṭigocare bhavati // \dehlia <- Passt sa = saḥ, weil es sich auf padārtho m.? bezieht?
%sa sādhakasya dṛṣṭigocare bhavati // \nepal
%
%It arises in the range of sight of the practitioner.  
%------------------------------
%
%ayamevordhvalakṣyaḥ                       nāsikāyāḥ upari dvādaśāṃgulamūlaparyantaṃ dṛṣṭiḥ sthirā karttavyā / \edprint

%ayamevordhvalakṣyaḥ   athādholakṣaḥ       nāsikāyā upari dvādaśāṃgulaparyantaṃ dṛṣṭiḥ sthirā karttavyā / \pune
%ayamevordhvalakṣya // atha adholakṣyaḥ // nāsikāyā upari dvādaśaṃgulaparyaṃtaṃ dṛṣṭiḥ sthirā karttavyā // \nepal
%ayamevordhvalakṣyaḥ // atha adholakṣaḥ // nāsikāyā upari dvādaśaṃgulaparyaṃtaṃ dṛṣṭiḥ sthirā karttavyā // \dehlia
%
%This truly being the upward directed fixation (\textit{ūrdhvalakṣya}). Now the downward directed fixation object (\textit{adholakṣya}). One should stabilize the gaze within the circumference (\textit{paryanta}) of twelve \textit{aṅgula}s beyond the nose.
%
%
%------------------------------
%
%athavā nāsikāyā agre dṛṣṭiḥ sthirā karttavyā / \edprint
%athavā nāsikāyā agre dṛṣṭiḥ sthirā karttavyā / \pune
%athavā nāsikāyā agre dṛṣṭiḥ sthirā karttavyā // \nepal
%%athavā nāsikāyā agre dṛṣṭiḥ sthirā karttavyā // \dehlia
%
%Or one should stabilize the gaze onto the tip of the nose. 
%
%------------------------------
%lakṣadūyasya dṛḍhīkaraṇāt / dṛṣṭiḥ sthirā bhavati / \edprint
%lakṣadvayasya dṛṣṭīkaraṇāt / dṛṣṭiḥ sthirā bhavati / \pune
%lakṣadvayasya dṛdhīkaraṇāt dṛṣṭiḥ sthirā bhavati / \nepal 
%lakṣadvayasya dṛdhīkaraṇāt dṛṣṭiḥ sthirā bhavati / \dehlia
%
%The fixation becomes stable due to firm exercise of the twofold fixation points (\textit{lakṣa}). 
%
%------------------------------
%pavanaḥ sthiro bhavati / \edprint
%pavanaḥ sthiro bhavati / \pune
%pavanaḥ sthiro bhavati / \nepal
%pavanaḥ sthiro bhavati / \dehlia
%
%The breath becomes stable. 
%
%------------------------------
%āyurvarddhate / \edprint
%āyurvarddhate / \pune
%āyurvardhate / \nepal
%āyurvardhate / \dehlia
%
%Vitality increases. 
%
%------------------------------
%etaddūyamapi bāhyalakṣyameva bhavati bāhyāṃtara ākāśe śūnyalakṣyaṃ karttavyaḥ / \edprint
%etad dvayamapi bāhyalakṣyameva bhavati bāhyābhyaṃtare ākāśe cet śūnyalakṣyaṃ karttavyaḥ / \pune
%etatadvayamevavākhyalakṣamapi kathyate // bāhyo bhyaṃtaraṃ ākāśavat śūnyalakṣyaḥ karttavyaḥ / \nepal
%etatadvayamevabāhyalakṣamapi kathyate // bāhyo bhyaṃtaraṃ ākāśavat // śūnyalakṣyaḥ karttavyaḥ / \dehlia
%
%This outward directed fixation is twofold as well. 
%
%------------------------------
%jāgraddaśāyāṃ calanadaśāyāṃ bhojanadaśāyāṃ sthitikāle sarvasthāne śūnyasya dhyānakāraṇāt // \edprint
%jāgraddaśāyāṃ calanadaśāyāṃ bhojanaṃ daśāyāṃ sthitikāle sarvasthāne śūnyasya dhyānakāraṇāt // \pune
%jāgraddaśāyāṃ cakabadaśāyāṃ bhojanadaśāyāṃ sthitikāle sarvvasthāne śūnyasya dhyānakaraṇāt maraṇatrāso na bhavati// \nepal
%jāgraddaśāyāṃ calanadaśāyāṃ bhojanadaśāyāṃ sthitikāle sarvvasthāne śūnyasya dhyānakaraṇāt maraṇatrāso na bhavati// śūnya \dehlia
%
%The fear of dying does not arise due to the exercise of meditation on the void at all places during ones life - while eating, moving and waking. 
%
%------------------------------
%idānīṃ rājayogayuktasya śarīre yaccihnaṃ tat kathyate / \edprint
%idānīṃ rājayogayuktasya puruṣasya yaccharīracihnaṃ kathyate / \pune
%idānīṃ rājayogayuktasya puruṣasya yaccarīracihnaṃ tat kathyate / \nepal
%idānīṃ rājayogayuktasya puruṣasya yaccarīracihnaṃ tat kathyate / \dehlia
%
%Now it is said that a characteristic of the body of the person is being endowed with the royal yoga. 
%
%%------------------------------
%tatsarvatra pūrṇo bhavati / \edprint
%tatsarvatra pūrṇā bhavati / \pune
%   sarvvatra pūrṇo bhavati / \nepal
%   sarvvatra pūrṇo bhavati  \dehlia
%%------------------------------
%
%pṛthivyāḥ dūre tiṣṭati / \edprint
%pṛthivyāḥ hare tiṣṭati / \pune
%pṛthivyāḥ dūre tiṣṭati / \nepal
%pṛthivyāḥ dūre tiṣṭati / \dehlia
%%------------------------------
%
%pṛthivyāṃ vyāpya tiṣṭhati / \edprint
%pṛthivyāpya      tiṣṭhati / \pune
%pṛthvāṃ vyāpya tiṣṭhati / \nepal
%pṛthvīṃ vyāpya tiṣṭhati / \dehlia
%------------------------------
%yasya janmamaraṇe na staḥ sukhaṃ na bhavati / \edprint
%yasya janmamaraṇe na staḥ sukhaṃ na bhavati / \pune
%yasya janmamaraṇe na staḥ sukhaṃ na bhavati / \nepal
%yasya janmamaraṇe na staḥ sukhaṃ na bhavati / \dehlia
%
%%------------------------------
%
% \om                 \edprint
% \om                 \pune
%duḥkhaṃ na bhavati / \nepal
%duḥkham na bhavati / \dehlia
%%------------------------------
% \om               \edprint
%kūlaṃ na bhavati / \pune
%kūlaṃ na bhavati / \nepal
%kūlaṃ na bhavati / \dehlia
%------------------------------
% \om               \edprint
%śītalaṃ na bhavati / \pune
%śīlaṃ na bhavati / \nepal
%śīlaṃ na bhavati / \dehlia
%------------------------------
% \om                 \edprint
%sthānaṃ na bhavati / \pune
%sthānaṃ na bhavati / \nepal
%sthānaṃ na bhavati / \dehlia
%------------------------------
% \om                                                                            \edprint
%asya siddhasya manomadhye īśvarasaṃbaṃdhī prakāśo  niraṃtaraṃ pratyakṣo bhavati  \pune
%asya siddhasya manomadhye īśvarasaṃbaṃdhī prakāśaḥ niraṃtaraṃ pratyakṣa bhavati  \nepal
%asya siddhasya manomadhye īśvarasaṃbaṃdhi prakāśaḥ niraṃtaraṃ pratyakṣo bhavati  \dehlia
%------------------------------
%
%sa ca prakāśo na śīto na coṣṇo na śveto na pīto bhavati / \edprint
%sa ca prakāśo na śīto na coṣṇo na śveto na pīto bhavati / \pune
%
%------------------------------
%
%tasya na jātirna kiñciccihnam / edprint
%tasya na jātirna kiñciccihnaṃ  \pune
%
%------------------------------
%ayaṃ ca niṣkalo niraṃjanaḥ alakṣyaśca bhavati \edprint 
%ayaṃ ca niṣkalo niraṃjanaḥ alakṣyaśca bhavati \pune
%
%
%------------------------------
%
%atha ca phalaṃdvade na kāminyāderyasyecchā na bhavati // \edprint
%atha ca phalacaṃda na  kāminyāderyasyochā na bhavati  \pune
%
%
%------------------------------
%anyad rājayogasya cihnaṃ kathyate । yasya rājyādilābhe'pi phalalābho na bhavati। hānāvapi manomadhye duḥkhaṃ na bhavati। atha ca tṛṣṇā na bhavati। atha ca kasmin padārthasyoparyanicchā na bhavati। kasmin padārthe manasonurāgo na bhavati। ayamapi rājayogaḥ kathyate। athacaḥ yasya manaḥ munividvatpuruṣeṣu metrai ca samaṃ bhavati। dṛṣṭiśca samā bhavati। sakalapṛthvīmadhye gamanavataḥ sukhabhogavataḥ yasya manasi karttṛtvābhimāno nāsti। atha ca lokamadhye gamanavataḥ sukhabhogavataḥ yasya manasi karttṛtvābhimāno nāsti। atha ca lokamadhye kartṛtvaṃ na jñāpayati। sopi rājayogaḥ kathyate।।
%navīnāni paṭṭasūtramayadhṛtāni vasrāṇi athavā jīrṇāni chidrāṇi dhṛtāni kastūrīcandanalepairvā kardamalepena yasya manasi harṣaśokau na staḥ। sa evātra tiṣṭhati। yasya janmamaraṇe na staḥ sukhaṃ na bhavati। kulaṃ na bhavati śīlaṃ na bhavati। sthānaṃ na bhavati। rājayogaḥ naramadhye atha ca vanamadhye yuddhe saṃgrāma madhye vā yasya manaḥ bhayapūrṇaṃ vā na bhavati। sopi rājayogaḥ kathyate।।
%
\endprose
\medskip
\endnumbering
\end{sanskrit}

\bigskip
\clearpage
\chapter{Translation of the \textit{Yogatattvabindu}}

\section*{\centerline{A Drop of the [Oceanic] Reality of Yoga}}


\beginnumbering
\pstart
\centerline{\textrm{\small{[Introduction]}}}
\pend

\bigskip

\numberlinefalse \noindent Homage to Śrī Gaṇeśa. Now the methods of \textit{rājayoga} are laid down. \par 
\numberlinefalse \noindent \par This is the result of \textit{rājayoga}\footnote{This statement seems unconnected to the definition of
  \textit{rājayoga} that follows.}: \textit{Rājayoga} is that by which long-term durability of the body arises even amongst manifold royal pleasures even amongst the manifold royal entertainments
and spectacle. This truly is \textit{rājayoga}. Of this [\textit{rājayoga}] these are the varieties:  1. Yoga of [mental] action (\textit{kriyāyoga}), 2. Yoga of knowledge (\textit{jñānayoga}), 3. Yoga of wandering (\textit{caryāyoga}), 4. Yoga of force (\textit{haṭhayoga}), 5. Yoga of deeds (\textit{karmayoga}), 6. Yoga of absorption (\textit{layayoga}), 7. Yoga of meditation (\textit{dhyānayoga}), 8. Yoga of mantras (\textit{mantrayoga}), 9. Yoga of fixation objects (\textit{lakṣyayoga}), 10. Yoga of mental residues (\textit{vāsanāyoga}), 11. Yoga of Śiva (\textit{śivayoga}), 12. Yoga of Brahman (\textit{brahmayoga}), 13. Yoga of non-duality (\textit{advaitayoga}), 14. Yoga of completion (\textit{siddhayoga}) 15. Yoga of kings (\textit{rājayoga}). These are the fifteen \textit{yoga}s.\footnote{At the current stage of research it is not clear if this list is a later addition by another scribe or, if indeed it originally stems from Rāmacandra. The list suggests a text following the order of \textit{yoga}s according to this list. However, the order and even the designation of some of the \textit{yoga}s given in the list is absent.} 

\bigskip

\pstart
\centerline{\begin{english}\textrm{\small{[Description of \textit{kriyāyoga}]}}\end{english}}
\pend
\bigskip

Now the characteristic of the Yoga of [mental] action (\textit{kriyāyoga}) is described.
\bigskip

\leftskip2em This Yoga is liberation through [mental] action. It bestows success (\textit{siddhi}) in ones own body. Each wave the mind creates at the beginning of an action, of all those one shall withdraw oneself. Then \textit{kriyāyoga} arises. 

\bigskip

\leftskip2em Patience, discrimination, equanimity, peace, modesty, desireless: The \textit{yogī} who is endowed with these means is said to be a \textit{kriyāyogī}. 
\bigskip

\leftskip2em Envy, selfishness, cheating, violence and intoxication, pride, lust, anger, fear, laziness, greed, error and impurity. 
\bigskip

\leftskip2em Attachment and aversion, indignation and idleness, impatience and dizzyness:  Whoever does not possess these is called a \textit{kriyāyogī}. 
\bigskip

[If] patience, discrimination, equanimity, peace, contentment etc. are generated in his mind [then] alone he is called a \textit{yogī} of many actions (\textit{bahukriyāyogī}). Fraud, illusion, property, violence, craving, envy, ego, anger, anxiety, shame, greed, error, impurity, attachment, aversion, idleness, heterodoxy, false view, affection of the senses, sexual desire: He who diminishes these from day to day in is mind, he alone is called a \textit{yogī} of many actions (\textit{bahukriyāyogī}). 

\bigskip
\pstart
\centerline{\begin{english}\textrm{\small{[Varieties of \textit{rājayoga}: Siddhakuṇḍalinīyoga and Mantrayoga]}}\end{english}}
\pend
\bigskip

Now varieties of \textit{rājayoga} will be described. Which are these? One is \textit{siddhakuṇḍalinīyoga} [and one] is \textit{mantrayoga}. These two \textit{rājayoga}s are described [in the following].

At the location of the root-bulb exists one major vessel in the form of energy (\textit{tejas}). This single vessel reaches into these openings which are \textit{iḍā}, \textit{piṅgalā} and \textit{suṣumnā}. On the left side is the \textit{iḍā}-channel being a resemblance of the moon. On the right side is the \textit{piṅgalā}-channel being a resemblance of the sun. Within the middle path is a lotuspond being very subtle. [It is] made from a web of light [and it] shines like a thousand lightnings. She is the bestower of enjoyment and liberation. While abiding in (\textit{satyāṃ}) her (\textit{asyāṃ}) knowledge arises [to the point of which] the person becomes all-knowing. The means for the genesis of knowledge in the central channel \textit{suṣumnā} will now be described.

\bigskip
\pstart
\centerline{\begin{english}\textrm{\small{[Description of the first Cakra]}}\end{english}}
\pend
\bigskip

At the beginning [of the central channel?] exists the root-\textit{cakra} having four petals. The first \textit{cakra} of support (\textit{ādhāra}) is at the anus, [it] is red-colored, [it] has Gaṇeśa as its deity, [he] is success, intelligence and power, [and has] a rat as [his] mount, the Ṛṣi [of it] is Kūrma, [its \textit{mudrā}] is the \textit{mudrā} of contraction (\textit{ākuñcamudrā}), [its] vitalwind is \textit{apāna}, in the four petals [of it resides] \textit{rajas}, \textit{sattva}, \textit{tamas} and mindstuffs?!(\textit{manāṃsi}) [symbolized by the syllables] ``vaṃ'', ``śaṃ'', ``ṣaṃ'' and ``saṃ'', in the middle [of it] is a triangle, in the middle [of the triangle] is a trident, and \textit{kāmapīṭha} in the shape of a triangle. In the middle of this seat (\textit{pīṭha}) exists a single form having the shape of a flame. Trough the practice of meditation on this form (\textit{mūrti}) the whole literature, all \textit{śāstras}, all poems, dramas etc. appear in the mind of the person without (prior) learning.

\bigskip
\pstart
\centerline{\begin{english}\textrm{\small{[Description of the second Cakra]}}\end{english}}
\pend
\bigskip

Now the second (will be described). The \textit{svādhiṣṭānacakra} having six petals is located at the seat known as \textit{uḍḍīyāṇa}. In its middle exists extremely red glow. The adept becomes very handsome through meditation on it. He becomes the most beloved of the virgins. The vital force increases from day to day.

\bigskip
\pstart
\centerline{\begin{english}\textrm{\small{[Description of the third Cakra]}}\end{english}}
\pend
\bigskip


The third, a lotus with ten petals exists at the location of the navel. In its middle exists a \textit{cakra} with five angles. In its middle is a single (divine) form. It is not possible to describe her shine with speech (lit. with the tongue). Through the execution of meditation on this (divine) form the body of the person (the meditator) is going to be strong.

\bigskip
\pstart
\centerline{\begin{english}\textrm{\small{[Description of the fourth Cakra]}}\end{english}}
\pend
\bigskip
A fourth lotus having twelve petals exists in the middle of the heart. Due to being made of (such an) intense light (the fourth lotus) is not in the range of sight. In its middle exists a lotus having eight petals which is facing downward. It is said that in its middle is the place of the vitalwind \textit{prāṇa} (and) in the middle (of) the eight-petalled lotus is a pericarp (\textit{karṇikā}) in the form of a \textit{liṅga}. The technical designation of her is \textit{kalikā}. In the middle of this \textit{kalikā} exists a single thumb-sized (divine) figurine (\textit{puttalikā}) being similiar to a ruby-gem in color. Her technical designation is embodied soul (\textit{jīva}). Not even with a thousand tongues it is possible to talk about her nature and her power. ``Because of the exercise of meditation on this form the inhabitants of the universe (which are) Humans, Gandharvas, Kinnaras, Guhyakas, Vidyādharas and (their) females, in the heavenly world, underworld and open space are obedient to the will of the practicing person.'', is said here.
\bigskip
\pstart
\centerline{\begin{english}\textrm{\small{[Description of the fifth Cakra]}}\end{english}}
\pend
\bigskip
Now (follows the description of) the fifth lotus having sixteen petals (which) exists at the location of the throat. In its  middle exists a single person which shines like a thousand moons. Because of the exercise of meditation on this person all incurable diseases vanish. The person lives up to 1001 years.
\bigskip
\pstart
\centerline{\begin{english}\textrm{\small{[Description of the sixth Cakra]}}\end{english}}
\pend
\bigskip
Now (there) exists the sixth \textit{cakra} named \textit{ājñā}. This two-petalled \textit{cakra} is situated inbetween the eyebrows. In its middle exists a certain object (\textit{kiṃcid vastu}) being a form of blazing fire without parts, not being female not being male. Because of the exercise of meditation on it the body of the person becomes non-aging and immortal.
\bigskip
\pstart
\centerline{\begin{english}\textrm{\small{[Description of the seventh Cakra]}}\end{english}}
\pend
\bigskip
Now the seventh cakra having 64 petals and being full of nectar exists in the middle of the palate. Being endowed with superabundant beauty, very bright in its middle [and] red in color exists a single pericarp named ``uvula'' (\textit{ghāṇṭikā}). In its middle is a (certain) place (\textit{bhūmi}).
In its middle exists a flow of nectar like a river (\textit{amṛtādhārāsravantī}), appearing from the digit of the moons disc.
Because of the exercise of meditation on this digit death does not come near him. Due to uninterrupted meditation the stream of nectar flows. The appearances of emaciation (\textit{kṣayaroga}), fever due to disordered bile (\textit{pittajvara}), heartburn (\textit{hṛdayadāha}), head-disease (\textit{śiroroga}) and tongue insensibility (\textit{jihvāyajaḍa}) vanishes. Eaten venom doesn't trouble him, too. If one is here the mind becomes stable.
\bigskip
\pstart
\centerline{\begin{english}\textrm{\small{[Description of the eighth Cakra]}}\end{english}}
\pend
\bigskip
Now the eighth \textit{cakra}, which has one hundred petals. It exists at the aperture of Brahman. ``The (divine) seat of Jālandhara'' is the designation of the lotus.
(It is) a place of the accomplished person (\textit{siddhapuruṣasya}). In its middle exists some single (divine) form of the person (\textit{puruṣa}) being a streak(\textit{rekhā}) having the form of smoke and fire. Of this (divine form) exists no end, nor a beginning. Due to the exercise of meditation on this (divine) form both coming and going of the person (\textit{puruṣa}) in space occurs. Affliction from the earth-element does not arise even if he is situated in the middle of the earth. He constantly perceives everything (in the material world) and he becomes separated (from the material world). The force of life increases eminently.
\bigskip
\pstart
\centerline{\begin{english}\textrm{\small{[Description of the ninth Cakra]}}\end{english}}
\pend
\bigskip
Now the differentiations of the ninth \textit{cakra} are explained. The designation of it is ``the \textit{cakra} of the great void'' (\textit{mahāśūnyacakra}). Above that there is no other. Therefore it is declared to be the \textit{cakra} of the great perfection (\textit{mahāsiddhacakra}). A(nother) name of it is ``(divine) seat of Pūrṇagiri''. In the middle of the \textit{mahāśūnyacakra} exists one lotus facing upward, very red in color with a thousand petals - an abode of brilliance and wholeness. Whose fragrance is not in range of mind and speech. In the middle of this lotus exists one pericarp having the shape of a triangle. In the middle of the pericarp exists one seventeenth digit in the shape of a immaculé form. There exists a light of the digit shining like a thousand suns. (But) excessive heat is not arising. (Also it is) shining like a thousand moons (but) excess of cold is not arising. Due to the exercise of meditation upon the digit suffering does not arise in the mind of the practitioner (anymore). Above that is the place of infinite supreme bliss. There above exists \textit{śakti}. Being designated as such she is one single digit. Due to the exercise of meditation on this part the person gets whatever he wishes for. Due to the exercise of meditation on this part the person gets whatever he wishes for. He is furnished with royal pleasure and enjoyment.(Even for him who is) amusing oneself amongst women, and watching musical pleasures, the \textit{kāla} of the person grows daily like the \textit{kalā} of the moon in the bright half of the month. His body is not affected by merit and sin. Due to uninterrupted meditation the power of the light of the innate nature arises. He sees remotely located objects as if they'd be near. 
\endnumbering
\end{document}


%Große Fragen an das Kolloquium:

%Devanāgarī bitte checken nach geöffneten Sandhis -> weil input ist in IAST und manchmal vergesse ich, dass der Output Devanagari sein soll
%check for falsche Sandhis
%Vereinfachung des kritischen Apprarates -> atitejomayatvād dṛṣti... und atitejomayatvāt dṛṣṭi 3 zu 1.... bla bla diese Dinge 

%%% Local Variables:
%%% mode: LaTeX
%%% TeX-parse-self: t
%%% TeX-auto-save: t
%%% TeX-PDF-mode: t
%%% TeX-engine: xetex
%%% eval: (set-input-method "sanskrit")
%%% eval: (setq TeX-command-default "xelatex")
%%% eval: (TeX-fold-mode)
%%% TeX-master: t
%%% End:
